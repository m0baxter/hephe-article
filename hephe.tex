
\documentclass[aps, pra, reprint, groupedaddress, amsfonts,
               amsmath, amssymb, showpacs, nofootinbib]{revtex4-1}

\usepackage{microtype}
\usepackage{graphicx}
\usepackage{epstopdf}
\usepackage[utf8]{inputenc}
\usepackage[T1]{fontenc}
\usepackage[usenames,dvipsnames]{xcolor}
\usepackage{hyperref}
\usepackage{braket}

\begin{document}

\title{Ceci n'est pas un titre}
\author{Matthew Baxter}
\email[]{baxterma@yorku.ca}
\author{Tom Kirchner}
\email[]{tomk@yorku.ca}
\affiliation{Department of Physics and Astronomy, York University, Toronto, Ontario, Canada, M3J 1P3}
\author{Eberhard Engel}
\affiliation{Center for Scientific Computing, J.W. Goethe-Universit\"{a}t, D-60438 Frankfurt/Main,
             Germany}
\date{\today}

\begin{abstract}

   Theoretical total cross section results for all ionization/capture processes in the
   He\textsuperscript{+}-He collision system are presented in the approximate impact energy range
   10-1000~keV/amu. Calculations were performed with in the framework of time-dependent spin-density
   functional theory. The Krieger-Li-Iafrate approximation was used to determine an accurate 
   exchange-correlation potential in the exchange-only limit. The results of two models, one where
   electron translation factors are ignore and another where partial electron translation factors are
   included, have been compared with available experimental data as well as a selection of previous
   theoretical calculations.

\end{abstract}

\pacs{31.15.ee, 34.50.Fa, 34.10.+x, 34.70.+e (Maybe\dots)} %nnnote: probably change this

\maketitle

\begin{section}{Introduction \label{sec:intro}}

   Atomic collision systems are often explored as testbeds for few-electron quantum dynamics. The
   general collision system carries active electrons on both the target and projectile. The simplest
   generic collision system consists of target with two electrons and a single electron on the
   projectile. A prototypical example from this class of problem is the He\textsuperscript{+}-He system.

   A variety of charge transfer processes are possible. These can be broadly categorized into those that
   involve one active electron:
   %
   \begin{equation} \label{eq:tpi111}
      \mathrm{He}^+ + \mathrm{He} \rightarrow \mathrm{He}^+ + \mathrm{He}^+ + e^-
   \end{equation}
   %
   \begin{equation} \label{eq:tpi120}
      \mathrm{He}^+ + \mathrm{He} \rightarrow \mathrm{He} + \mathrm{He}^+,
   \end{equation}
   %
   \begin{equation} \label{eq:tpi201}
      \mathrm{He}^+ + \mathrm{He} \rightarrow \mathrm{He}^{2+} + \mathrm{He} + e^-,
   \end{equation}
   %
   two active electrons:
   %
   \begin{equation} \label{eq:tpi012}
      \mathrm{He}^+ + \mathrm{He} \rightarrow \mathrm{He}^+ + \mathrm{He}^{2+} + 2e^-
   \end{equation}
   %
   \begin{equation} \label{eq:tpi102}
      \mathrm{He}^+ + \mathrm{He} \rightarrow \mathrm{He}^{2+} + \mathrm{He}^+ + 2e^-
   \end{equation}
   %
   \begin{equation} \label{eq:tpi021}
      \mathrm{He}^+ + \mathrm{He} \rightarrow \mathrm{He} + \mathrm{He}^{2+} + e^-,
   \end{equation}
   %
   and three active electrons:
   \begin{equation} \label{eq:tpi003}
      \mathrm{He}^+ + \mathrm{He} \rightarrow \mathrm{He}^{2+} + \mathrm{He}^{2+} + 3e^{-}.
   \end{equation}
   %
   Additionally there is the channel where no charges are transferred and the two channels that result
   in the production of He\textsuperscript{-}. From the perspective of theory, the later is unphysical
   due to the finite lifetime of even the most stable species of the negative helium
   ion~\cite{neghelife}. From the point of view of experiment He\textsuperscript{-} production
   may be controlled for~\cite{metahe} and cross sections can even be made negligible~\cite{neghe-neg}.

   Total cross sections for the processes described in Eqs.~\eqref{eq:tpi111}-\eqref{eq:tpi003} have
   been calculated within a spin-density functional theoretic~\cite{td-spindep} framework, a
   generalization of time-dependent density functional theory~\cite{tddft, ullrich} to spin-dependent
   systems. The spin-polarized nature of the system accentuates the importance of exchange effects
   which in turn necessitates an accurate exchange potential. An exploration of a procedure for
   calculating such potential comprises the bulk of this work.

   Experimental results for the various outcome channels over a large range of impact
   energies~\cite{BS58, HN78, HSE78, dCFdP88, DT-88, Dub-89, ASL91, FTFHLP-95, SSMSM-11} provide a
   useful benchmark for this process. Results can also be compared with some of the theoretical work of
   other groups. These calculations employ a variety of methods including classical models like the
   over the barrier model~\cite{CC-07}, Bohr–Lindhard model~\cite{DYC-08, DLZ-12}, and the classical
   trajectory Monte Carlo method~\cite{GMZ17}. Quantum mechanical calculations have been carried out
   using the local plasma approximation~\cite{MMA03}, the independent event model~\cite{SM-03}, as well
   as a plethora of perturbative calculations~\cite{Mancev96, BOC05, Mancev-07, MG-10, NTC11, GG-12b,
   GAG15}, the majority of which focus on single capture to the projectile [Eq.~\eqref{eq:tpi120}].

   We will begin our discussion with an overview of some relevant aspects of time-dependent density
   functional theory in Sec.~\ref{sec:tddft}. This will be followed by a description of the method used
   to calculate a time-dependent exchange potential in Sec.~\ref{sec:xpot}. This section also includes
   some details of the two-center basis generator method which was used to solve for the one-particle
   density. The theory section closes with a description of the final-state analysis for extracting
   the carious outcome probabilities [Sec.~\ref{sec:probs}]. The results of our calculations are
   presented in Sec.~\ref{sec:disc}. Finally conclusions drawn from these studies are offered in
   Sec.~\ref{sec:conc}.

   Atomic units ($\hbar = m_e = e = 1$) are used unless stated otherwise.

\end{section}

\begin{section}{Theory \label{sec:theory}}

   \begin{subsection}{TDDFT \label{sec:tddft}}

      A system of $N$ particles may be described by an $N$-particle wave function $\Psi(t)$ whose
      evolution is governed by the time-dependent Shr\"{o}dinger equation (\textsc{tdse})
      %
      \begin{equation} \label{eq:tdse}
         i \frac{\mathrm{d} \Psi(t)}{\mathrm{d}t} = \hat{H}(t),
      \end{equation}
      %
      with a Hamiltonian $\hat{H}$ which may be written as
      %
      \begin{equation} \label{eq:ham}
         \hat{H}(t) = \hat{T} + \hat{V}_\mathrm{ee} + \hat{V}_\mathrm{ext}(t),
      \end{equation}
      %
      where $\hat{T}$ is the kinetic energy, $\hat{V}_\mathrm{ee}$ are the two particle interactions,
      and $\hat{V}_\mathrm{ext}$ is a time-dependent, one-particle interaction potential.

      The computationally intensive two-body term $\hat{V}_\mathrm{ee}$ makes direct solutions of the
      \textsc{tdse} difficult. Time-dependent density-functional theory~\cite{tddft, ullrich} (TDDFT)
      offers a solution to this problem. By making use of the mapping between the one-particle density
      %
      \begin{equation} \label{eq:dendef}
         n(\mathbf{r},t) = N \sum\limits_{sigma} \int \mathrm{d}^3 r_2 \dots \mathrm{d}^3 r_N \,
                            \left| \Psi(\mathbf{x}_i,t) \right|^2
      \end{equation}
      %
      and the external potential $\hat{V}_\mathrm{ext}$, where $x_i = (\mathbf{r}_i, \sigma_i)$ label
      the position and spin of the $i$\textsuperscript{th} particle, guaranteed by the Runge-Gross
      theorem~\cite{rg_theorem, td-spindep} the interacting system may be mapped onto as system of
      non-interacting particles.

      This so-called Kohn-Sham system consists of $N$ spin-orbitals $\varphi_{j \sigma}$ which evolve
      according to the time-dependent Kohn-Sham equation
      %
      \begin{equation} \label{eq:tdks}
         i \frac{\partial}{\partial t} \varphi_{j\sigma} = \left( -\frac{\Delta}{2} +
               v^\sigma_\mathrm{\textsc{ks}}[n_\uparrow, n_\downarrow](\mathbf{r},t)
               \right) \varphi_{j\sigma}(\mathbf{r},t),
      \end{equation}
      %
      such that
      %
      \begin{equation} \label{eq:ksden}
         n(\mathbf{r},t) = \sum\limits_\sigma \sum\limits_{j=1}^{N_\sigma}
                           \left| \varphi_{j\sigma} \right|^2.
      \end{equation}
      
      In Eq.~\eqref{eq:tdks} $n_\sigma$ are the spin-up/down one-particle densities given by
      %
      \begin{equation} \label{eq:spinden}
         n_\sigma = \sum\limits_{j=1}^{N_\sigma} \left| \varphi_{j\sigma} \right|^2
      \end{equation}
      %
      and $v^\sigma_\textsc{ks}$ is the Kohn-Sham potential. The potential may be separated into several
      simpler objects
      %
      \begin{equation} \label{eq:vks}
         v^\sigma_\mathrm{\textsc{ks}}[n_\uparrow, n_\downarrow] = v_\mathrm{ext} + v_\mathrm{H}[n]
                                                   + v^\sigma_\mathrm{xc}[n_\uparrow, n_\downarrow].
      \end{equation}
      %
      The first term in this expression is the external potential, which is essentially the same as the
      potential $\hat{V}_\mathrm{ext}$ of Eqs.~\eqref{eq:ham}. For the He\textsuperscript{+}-He
      collision system considered in this work it may be written, making use of the semi-classical
      approximation,
      %
      \begin{equation} \label{eq:hephe-ext}
         v_\mathrm{ext}(\mathbf{r},t) = -\frac{2}{r} 
         - \frac{2}{\left| \mathbf{r} - \mathbf{R}(t) \right|},
      \end{equation}
      %
      where $\mathbf{R}(t) = (b,0,V t)$ is the straight-line trajectory of the projectile with velocity
      $V$ and impact parameter (distance of closest approach) $b$.
      
      The next term is the Hartree screening potential
      %
      \begin{equation} \label{eq:vh}
         v_\mathrm{H}(\mathbf{r},t) = \int \frac{n(\mathbf{r}^\prime, t)}
            {\left| \mathbf{r} - \mathbf{r}^\prime\right|} \, \mathrm{d}^3 r^\prime.
      \end{equation}
      %
      The last term is the exchange-correlation potential which encodes the complicated
      electron-electron interaction potential into the language of the non-interacting system. For
      convenience this is often further broken down into separate exchange and correlation potentials
      %
      \begin{equation} \label{eq:vxc}
         v^\sigma_\mathrm{xc} = v^\sigma_\mathrm{x} + v^\sigma_\mathrm{c}.
      \end{equation}

      Splitting $v_\mathrm{xc}$ into an exchange and correlation part facilitates the application of the
      x-only approximation where the correlation potential is taken to be zero ($v^\sigma_\mathrm{c} =
      0$). Such a model which ignores dynamic correlation is usually referred to as an independent
      electron model (\textsc{iem}). Within this approximation $v^\sigma_\mathrm{x}$ may be determined
      exactly via the optimized potential method~\cite{opm1, opm2, tdopm}. The complexity of the
      \textsc{opm} make it difficult prohibitively difficult to implement in general. As a secondary
      option one my instead make use of the Krieger-Li-Iafrate approximation~\cite{klieq, tdkli1, tdkli2}
      (\textsc{kli}). In many situations potentials generated using \textsc{kli} are numerically
      indistinguishable from those produced by the full \textsc{opm}. The success of the \textsc{kli} is
      due to the fact that it preserves several properties of the exact potential. In particular
      \textsc{kli} maintains the correct asymptotics
      %
      \begin{equation} \label{eq:asymp}
         \lim\limits_{r \rightarrow \infty} v_\mathrm{x}^\sigma (\mathbf{r}) = -\frac{1}{r}.
      \end{equation}
   
   \end{subsection}

   \begin{subsection}{The exchange potential \label{sec:xpot}}

      The one-particle density was determined by solving Eq.~\eqref{eq:tdks} using the two-center basis
      generator method~\cite{tcbgm} (\textsc{tc-bgm}). As mentioned above this relies upon the
      specification of an exchange-correlation potential. While the correlation potential may be
      ignored, that is the x-only approximation may be used (with some understanding of the
      consequences), an accurate exchange-potential is essential for a precise description of the
      He\textsuperscript{+}-He collision system. The spin polarized nature of the system, which
      emphasizes exchange effects. To this end the \textsc{kli} approximation to the \textsc{opm} was
      employed in the calculation of $v^\sigma_\mathrm{x}$.

      The ground-state density functional theory (\textsc{dft}) scheme of~\cite{diamol} has been adapted
      to calculate a time-dependent exchange potential. At any instant of time, $t$, the
      He\textsuperscript{+}-He system may be regarded as a diatomic quasi-molecule with an internuclear
      distance $R_\mathrm{int}(t) = \sqrt{b^2 + z(t)^2}$, where $b$ and $z$ are the impact parameter and
      $z$ position of the projectile as described below Eq.~\eqref{eq:hephe-ext}. If at each time-step
      of the \textsc{tc-bgm} the time-dependent Kohn-Sham orbitals, $\varphi_{\sigma j}(\mathbf{r},t)$
      are fed into the \textsc{kli} functional an exchange-potential,
      $v^{\sigma}_\mathrm{x}[\varphi_{\sigma j};t]$, may be calculated at each $t$, effectively
      giving one the time-dependent exchange-potential $v^{\sigma}_\mathrm{x}[\varphi_{\sigma j}](t)$.
      The restriction of the \textsc{kli} scheme to systems of cylindrical symmetry complicates this
      processes, in general $\varphi_{j \sigma}$ will not be forced to exhibit any specific symmetry.
      
      In order to detail a solution to the symmetry problem a more thorough description of the
      \textsc{tc-bgm} is necessary. Within the \textsc{tc-bgm} the Kohn-Sham orbitals are represented in
      the non-orthogonal basis
      %
      \begin{equation} \label{eq:bgmexp}
         \varphi_{\sigma j}(\mathbf{r},t) = \sum\limits_{c \in \{P, T\}} \sum\limits_{k, L}
                               d_{c k L}^{\sigma j}(t) \tilde{\chi}^{L}_{c k}(\mathbf{r},t),
      \end{equation}
      %
      where
      %
      \begin{equation} \label{eq:etfbasis}
         \tilde{\chi}^{L}_{ck}(\mathbf{r}),t =
            \begin{cases}
               e^{i \mathbf{v}_T \cdot \mathbf{r}} {\chi}^{L}_{c k}(\mathbf{r},t) & c = T, \\[2ex]
               e^{i \mathbf{v}_P \cdot \mathbf{r}} {\chi}^{L}_{c k}(\mathbf{r},t) & c = P,
            \end{cases}
      \end{equation}
      %
      which are the basis functions with electron translation factors (\textsc{etf}) included. The basis
      functions themselves are given by
      %
      \begin{equation} \label{eq:bgmbasis}
         \chi^{L}_{ck} (\mathbf{r},t)
         = W_P( \mathbf{r},t, \epsilon_P)^L \chi^{0}_{ck} (\mathbf{r}),
      \end{equation}
      %
      with
      \begin{equation}
         W_P (\mathbf{r},t,\epsilon_P)
         = \frac{1 - e^{-\epsilon_P|\mathbf{r}_T - \mathbf{R}(t)|}}{|\mathbf{r}_T - \mathbf{R}(t)|},
      \end{equation}
      %
      where $\mathbf{r}_T$ represents the position vector with respect to the target center.
      
      In Eq.~\eqref{eq:bgmbasis} the functions $\chi^{0}_{ck}$ are the ground-state orbitals for the
      target helium atom ($c = T$) and the projectile helium ion ($c = P$). Additional states that
      generated by a target potential operator are possible however in order to keep the number
      of states in the basis to a minimum and simplify the description only the projectile states are
      included. This simplification has proved sufficient in the past~\cite{bgm-rev}. The remaining
      regularizer is set to $\epsilon_P = 1$.

      It is clear from the above description that only the basis states corresponding s-type orbitals
      will make cylindrically symmetric contributions to the \textsc{ks}-orbitals. The simplest solution
      is to only feed the 1s contributions into the \textsc{kli} functional. While higher s-states are
      also admissible beyond 1s they will not be most occupied states meaning there inclusion will do
      little to improve the accuracy of the description of a given orbital. More explicitly the orbital
      will take the form (leaving out \textsc{etf}'s)
      %
      \begin{equation} \label{eq:1sonly}
         \varphi_{\sigma j}^{1s} = a^{\sigma j}_T \chi^{0}_{T1} + a^{\sigma j}_P \chi^{0}_{P1}.
      \end{equation}
      %
      The coefficients are the result of projecting onto the two-dimensional subspace spanned by the
      target and projectile 1s-states
      %
      \begin{equation} \label{eq:proj}
         \hat{P} = \sum\limits_{c_1, c_2 \in \{T,P\}} \tilde{S}^{-1}_{c_1 c_2}
                                                      \ket{\chi^{0}_{c_1 1}}
                                                      \bra{\chi^{0}_{c_2 1}}
      \end{equation}
      %
      with $\tilde{S}^{-1}_{c_1 c_2}$ the inverse of the overlap matrix
      %
      \begin{equation} \label{eq:overlap2d}
         \tilde{S}_{c_1 c_2} = \braket{ \chi^{0}_{c_1 1} | \chi^{0}_{c_2 1} }.
      \end{equation}
      %
      The coefficients are then
      %
      \begin{equation} \label{eq:coef}
         \begin{split}
            a^{\sigma j}_c & = \braket{ \chi^{0}_{c 1} | \hat{P} | \varphi_{\sigma j} } \\
                           & = \sum\limits_{c_1, c_2 \in \{T,P\}} \sum\limits_{k_1, k_2 = 1}^K
                               \sum\limits_{l_1, l_2 = 1}^L \tilde{S}^{-1}_{c c_1}
                               S_{c_1 \, k_1 \, l_1 }^{c_2 \, k_2 \, l_2 } d^{j \sigma}_{c_2 k_2 l_2},
         \end{split}
      \end{equation}
      %
      with
      %
      \begin{equation} \label{eq:overlap}
         S_{c_1 \, k_1 \, l_1 }^{c_2 \, k_2 \, l_2 } =
            \braket{ \chi^{l_1}_{c_1 k_1} | \chi^{l_2}_{c_2 k_2} }
      \end{equation}
      %
      the full overlap matrix.

      Returning to the question of the \textsc{etf}, working n the rotating center of mass frame the
      \textsc{etf}'s become
      %
      \begin{subequations} \label{eq:etf}
         \begin{equation} \label{eq:etfT}
            e^{i \mathbf{v}_T \cdot \mathbf{r}} =
             e^{\frac{i v_\mathrm{rel}}{2} (x \sin \theta - z \cos \theta)},
         \end{equation}
         %
         \begin{equation} \label{eq:etfP}
            e^{i \mathbf{v}_P \cdot \mathbf{r}} =
             e^{\frac{i v_\mathrm{rel}}{2} (z \cos \theta - x \sin \theta)},
         \end{equation}
      \end{subequations}
      %
      where $v_\mathrm{rel}$ is the relative velocity of the centers and $\theta = \arctan b/z$. If we
      now introduce a two-centered coordinated system, placing the foci at the two nuclear centers
      %
      \begin{equation} \label{eq:psc}
            \left\{
            \begin{array}{l}
               x = \frac{|R|}{2} \sqrt{(\xi^2 - 1)(1 - \eta^2)} \sin \phi, \\ [2ex]
               y = \frac{|R|}{2} \sqrt{(\xi^2 - 1)(1 - \eta^2)} \cos \phi, \\ [2ex]
               z = \frac{|R|}{2} \xi \eta,
            \end{array}
            \right.
      \end{equation}
      %
      it becomes clear that the portion of the \textsc{etf}'s containing only $x$ breaks cylindrical
      symmetry.

      Two obvious solutions present themselves. First, one may simply ignore the \textsc{etf}'s
      completely. This would amount to passing the orbitals described by Eq.~\eqref{eq:1sonly} into
      the \textsc{kli} functional. Alternately the symmetry breaking portions of the \textsc{etf}'s
      may be dropped. In this case the full 1s-only \textsc{ks}-orbital becomes
      %
      \begin{equation} \label{eq:1sonlyetf}
         \tilde{\varphi}_{\sigma j}^{1s} =
                       a^{\sigma j}_T e^{- \frac{i v_\mathrm{rel} z \cos \theta}{2}} \chi^{0}_{T1}
                     + a^{\sigma j}_P  e^{\frac{i v_\mathrm{rel} z \cos \theta}{2}}\chi^{0}_{P1}.
      \end{equation}
      %
      This will offer at least some of the correction provided by the full \textsc{etf}. Unfortunately
      As $z$ approaches zero the partial \textsc{etf} will tend to one, meaning that when the target and
      projectile are at their closest, the most active region of the collision, no \textsc{etf} will be
      present. 
      
      Regardless of which option is chosen it is important that $V_\mathrm{H}$ be determined with the
      same set of orbitals used in the calculation of $v_\mathrm{x}^\sigma$, preserving the precise
      cancellation of the self interaction present in the Hartree potential.

   \end{subsection}

   \begin{subsection}{Final-state analysis \label{sec:probs}}

      Of interest in any scattering problem is the probability of finding the system in some final state.
      If we represent the state of being considered as $\ket{f_1 f_2 f_3}$ and the initial state of the
      system propagated to some final time $t_f$ by $\ket{\varphi_{ \uparrow 1} \varphi_{\uparrow 2}
      \varphi_{\downarrow 1} (t_f)}$ the exclusive probability to find the system in the given final
      state at time $t_f$ will be given by
      %
      \begin{equation} \label{eq:finalProb}
         P_{f_1 f_2 f_3}(t_f) = \left| \braket{ f_1 f_2 f_3 | \varphi_{ \uparrow 1} \varphi_{\uparrow 2}
                                                \varphi_{\downarrow 1} (t_f) } \right|^2.
      \end{equation}
      %
      If one assumes that both the propagated and final states can be represented as a single Slater
      determinant then the probability in question becomes
      %
      \begin{equation} \label{eq:detProb}
         P_{f_1 f_2 f_3}(t_f) = \det \left[ \gamma_{f f^\prime}(t_f) \right],
      \end{equation}
      %
      where $\gamma_{f f^\prime}$ is the one-particle density matrix
      %
      \begin{equation} \label{eq:denmat}
         \gamma_{f f^\prime}(t_f) = \sum\limits_{\sigma} \sum\limits_{j = 1}^{N_\sigma}
                               \braket{f | \varphi_{\sigma j}(t_f)}
                               \braket{\varphi_{\sigma j}(t_f) | f^\prime}.
      \end{equation}
      %
      With the transition amplitudes $\braket{f | \varphi_{\sigma j}(t_f)}$ readily calculable from the
      dynamics. A model of this type which ignores the socalled functional correlations is consistent with
      an \textsc{iem} description.

      Alternatively, one could consider the probability to explicitly measure the states of some subset
      of the total number of particles. These so called inclusive probabilities can be expressed in
      terms of determinants of submatrices of the density matrix~\cite{inc-prob}.
 
      In the current problem we are interested those probabilities which correspond to the outcome
      channels of Eqs.~\eqref{eq:tpi111}-\eqref{eq:tpi003}. In such configurations we find $k$ particles
      on the projectile, $l$ particles in the continuum, and $3 - k - l$ on the target ($1\leq k \leq 3$
      and $1 \leq l \leq 3 - k$). The probabilities $p_{kl}$ may be calculated in terms of sums of
      inclusive probabilities to find a given number of particles in the bound states of the target and
      projectile by applying the machinery of~\cite{inc-prob} (see for example~\cite{incEx, mitsuko12,
      gerald15}).

   \end{subsection}

\end{section}

\begin{section}{Discussion \label{sec:disc}}

   \begin{figure}[t]
      \centering
      \includegraphics[width = \linewidth]{./images/crossPlots/HepHe-111.eps}
      \caption{Total Cross section for single ionization of the target.
               Theoretical: p\textsc{etf} (solid line), n\textsc{etf} (dotted line), and
                            \textsc{cdw-eis} of Miraglia \textit{et al}.~\cite{MG-10} (dashed line).
               Experimental: diamonds~\cite{Dub-89}, circles~\cite{FTFHLP-95}, and squares~\cite{DT-88}.
               \label{fig:cs111}}
   \end{figure}

   \begin{figure}[t]
      \centering
      \includegraphics[width = \linewidth]{./images/crossPlots/HepHe-201.eps}
      \caption{Total Cross section for single ionization of the projectile.
               Theoretical: p\textsc{etf} (solid line),n\textsc{etf} (dotted line), and
                            \textsc{ievm} of Sigaud \textit{et al}.~\cite{SM-03} (dashed line).
               Experimental: diamonds~\cite{Dub-89}. \label{fig:cs201}}
   \end{figure}

   \begin{figure}[t]
      \centering
      \includegraphics[width = \linewidth]{./images/crossPlots/HepHe-120.eps}
      \caption{Total Cross section for single capture to the projectile.
               Theoretical: p\textsc{etf} (solid line), n\textsc{etf} (dotted line), and
                            \textsc{cdbw-4b} (post form) of Ghanbari-Adivi \textit{et al}.~\cite{GAG15}
                            (dashed line).
               Experimental: diamonds~\cite{Dub-89} and circles~\cite{FTFHLP-95}. \label{fig:cs120}}
   \end{figure}

   \begin{figure}[t]
      \centering
      \includegraphics[width = \linewidth]{./images/crossPlots/HepHe-012.eps}
      \caption{Total Cross section for double ionization of the target.
               Theoretical: p\textsc{etf} (solid line) and n\textsc{etf} (dotted line).
               Experimental: diamonds~\cite{Dub-89}, circles~\cite{FTFHLP-95}, and squares~\cite{DT-88}.
               \label{fig:cs012}}
   \end{figure}

   \begin{figure}[t]
      \centering
      \includegraphics[width = \linewidth]{./images/crossPlots/HepHe-021.eps}
      \caption{Total Cross section for transfer ionization of the target.
               Theoretical: p\textsc{etf} (solid line) and n\textsc{etf} (dotted line).
               Experimental: diamonds~\cite{Dub-89} and circles~\cite{FTFHLP-95}. \label{fig:cs021}}
   \end{figure}
   \begin{figure}[t]
      \centering
      \includegraphics[width = \linewidth]{./images/crossPlots/HepHe-102.eps}
      \caption{Total Cross section for simultaneous single ionization of the target and projectile.
               Theoretical: p\textsc{etf} (solid line), n\textsc{etf} (dotted line), and
                            \textsc{ievm} of Sigaud \textit{et al}.~\cite{SM-03} (dashed line).
               Experimental: diamonds~\cite{Dub-89} and crosses~\cite{SSMSM-11}. \label{fig:cs102}}
   \end{figure}

   \begin{figure}[t]
      \centering
      \includegraphics[width = \linewidth]{./images/crossPlots/HepHe-003.eps}
      \caption{Total Cross section simultaneous double target and single projectile ionization.
               Theoretical: p\textsc{etf} (solid line), n\textsc{etf} (dotted line), and
               \textsc{ievm} of Sigaud \textit{et al}.~\cite{SM-03} (dashed line).
               Experimental: diamonds~\cite{Dub-89} and crosses~\cite{SSMSM-11}. \label{fig:cs003}}
   \end{figure}

   With the probabilities in hand the total cross sections may then be calculated from
   %
   \begin{equation} \label{eq:cross}
      \sigma_{kl} = 2 \pi \int_0^\infty b \, p_{kl}(b) \, \mathrm{d}b.
   \end{equation}

   In what follows all Results obtained by propagating the full \textsc{ks} orbitals in a potential
   generated from the 1s-only orbital of Eq.~\eqref{eq:1sonly} that includes no electron translation
   factors will be designated by n\textsc{etf}. Those obtained by an application of the same processes
   using the 1s-only orbital, with partial \textsc{etf}'s, of Eq.~\eqref{eq:1sonlyetf} will be referred
   to as p\textsc{etf}.
   
   Total cross sections for the processes described in
   Eqs.~\eqref{eq:tpi111}-\eqref{eq:tpi003} are presented in Figs.~\ref{fig:cs111}-\ref{fig:cs003}.
   Where available the results of the current work are compared with calculations of other groups. It
   should be noted that only those calculations that describe the system quantum mechanically were
   considered, that is to say works that employ approaches such as the classical trajectory Monte Carlo
   method are not included.

   We begin the discussion by considering single target ionization the process of Eq.~\eqref{eq:tpi111}.
   The results for this channel, $\sigma_{11}$ are presented in Fig.~\ref{fig:cs111}. Both n\textsc{etf}
   and p\textsc{etf} are in good agreement with experiment throughout the full range of impact energies.
   As impact energy ($E_P = \frac{1}{2} m_\textrm{He} V$) increases a slight gap opens between the two.
   This trend is as one would expect due to the fact the \textsc{etf}'s should become more relevant as
   the relative velocity between target and projectile increases. The slightly lower values for the
   p\textsc{etf} version above 200~keV/amu make it a better fit to experiment. The underestimation of
   both curves below the peak may be due to correspond almost exactly with the region where
   $\sigma_{02}$ results begin to rise above the experimental results [see Fig.~\ref{fig:cs120}].

   Also displayed in Fig.~\ref{fig:cs111} are the Continuum-Distorted-Wave Eikonal-Initial-State
   approximation (\textsc{cdw-eis}) results of Miraglia \textit{et al}.~\cite{MG-10}. These results seem
   to compliment the results of the present work through the majority of the impact energy range. One
   notable exception to this is the slightly higher cross section maximum however, as there is a fairly
   large spread in the experimental data around this region it is difficult to say which is more
   correct. The results of Miraglia \textit{et al}.\ also begin to diverge as approaches lower impact
   energies. This feature is likely due in large part the perturbative nature of \textsc{cdw-eis} which
   becomes less reliable as one decreases the impact energy.

   Next, we consider the results for $\sigma_{01}$ [Eq.~\eqref{eq:tpi201}] shown in Fig.~\ref{fig:cs201}.
   AS with the previous channel both n\textsc{etf} and p\textsc{etf} results are in good agreement with
   the experiment where it is available. Also continuing the trend seen in the $\sigma_{11}$ results
   both models begin essentially equal at low impact energies and diverge as $E_P$ increases. Both
   models begin to over estimate the data above the peak around 200~keV/amu. Once again the slightly
   lower p\textsc{etf} results are in better agreement with experiment. The slight unphysical structures
   in the curves below 40 keV/amu seem to correspond with the peeks of $\sigma_{00}$ channel (not
   pictured here). This issue will be discussed in greater detail below.

   These calculations have been compared with the independent event model (\textsc{ievm}) results of
   Sigaud \textit{et al}.~\cite{SM-03}. Their values seem to be in much better agreement with experiment
   in the high energy range then either n\textsc{etf} or p\textsc{etf} models. This can be easily
   explained the presence of correlation in these calculations. In particular, Sigaud \textit{et al}.\
   claim to capture the effects of antiscreening, the direct interaction between target and projectile
   electrons, which becomes increasingly import for projectile ionization processes at larger impact
   energies. As the results of the current work are those of an independent electron model
   (\textsc{iem}) they make no effort to capture any correlation effects.

   For the results of single electron capture to the target, the process of Eq.~\eqref{eq:tpi120}
   depicted in Fig.~\ref{fig:cs120}, both n\textsc{etf} and p\textsc{etf} models are essentiall
   identical. This is what one would hope for as they are in very good agreement with experiment in the
   entire range of impact energies. An explanation of the slight discrepancy between theory and
   experiment in the 50-150~keV/amu interval is offered by a comparison with the four-body Coulomb–Born
   distorted wave approximation (\textsc{cdbw-4b}) results of Ghanbari-Adivi
   \textit{et al}.~\cite{GAG15}. The correlation effects included in this model point to the slight rise
   in cross section being related to either the fact that we have employed an \textsc{iem} approximation
   or a failure of the partial \textsc{etf}.

   The later explanation may provide a more satisfying solution to this problem. One would expect that
   capture processes should be dominate by the contributions of slow and close collisions. The regions
   where the n/p\textsc{etf} models start to diverge from experiment is approximately the region where
   both models begin to diverge, that is the lowest energies where \textsc{etf}'s are important.
   Additionally they begin to agree with experiment once the cross sections begin to rapidly approach
   zero, for fast collisions. This would seem to be evidence that correct \textsc{etf}'s are important
   for capture processes (a fact that should be at least intuitively obvious).

   A few words should be spent addressing the choice for the theoretical calculation compared against.
   Unlike other channels there exist relatively large number works to select from that fit the criteria
   listed above. As the majority of these belong to a family of related perturbative
   models~\cite{Mancev96, BOC05, Mancev-07, MG-10, NTC11, GG-12b, GAG15} the latest, that of
   Ghanbari-Adivi \textit{et al}.\ was chosen.

   With the single electron processes taken care of double target ionization, Eq.~\eqref{eq:tpi012},
   the first of the two-electron processes will be considered next. The results for this channel are
   presented in Fig.~\ref{fig:cs012}. As with previous channels both n\textsc{etf} and p\textsc{etf}
   results appear to be very similar with a slight edge going to the p\textsc{etf} model's marginally
   lower results above 100~keV/amu. Both models seem to shift the peak in the cross section to higher
   impact energy than experiment would suggest is correct. As one would expect from an \textsc{iem}
   the two models exagerate double ionization. As there no previous works fitting the conditions for
   inclusion list earlier little else can be concluded from about the results of the present work.

   Another channel where the literature lacks a proper touchstone is that of transfer ionization\
   [Eq.~\eqref{eq:tpi021}]. The trends for $\sigma_{21}$ are very similar to those of $\sigma_{12}$.
   As with the previously discussed process both models are above experiment and shift the experimental
   peak to a higher impact energy. The only significant difference is that this is one of the few
   channels where the n\textsc{etf} model is in slightly better agreement with experiment then the
   p\textsc{etf}. The flattening of the curves below 20~keV/amu are an artifact of the \textsc{tc-bgm}
   becoming less reliable at the lowest impact energies.

   The last two-electron processes is for simultaneous single ionization of the target and projectile,
   Eq.~\eqref{eq:tpi102}. The results for our n\textsc{etf} and p\textsc{etf} models are presented in
   Fig.~\ref{fig:cs102}. These results both follow the trend of the data quite closely, arguably
   matching the peak in the experimental cross sections. This channel is one of the only two where
   the n\textsc{etf} model has as slight egde over the results of the p\textsc{etf}. Unfortunately they
   fall below experiment for the majority of the impact energy range shown.

   A comparison with the \textsc{ievm} of Sigaud \textit{et al}.\ explain this fact. As mentioned above
   the region above the cross section peak at around 150~keV/amu is the range where antiscreening
   dominates projectile ionization processes. Sigaud \textit{et al}.'s efforts to capture antiscreening
   see their results fall within experiment for their entire extent. Encouragingly, if one were to
   extend the curve of Sigaud \textit{et al}.\ it would seem to overlap with the results of the present
   work.

   Finally we will consider the sole three electron process, simultaneous double target and single
   projectile ionization [Eq.~\eqref{eq:tpi003}]. The results, presented in Fig.~\ref{fig:cs003}, again
   follow the general trends found in the previously discussed channels; overestimation of the cross
   section peak and a slightly better showing for the p\textsc{etf} model over the results of the
   n\textsc{etf} model. Unlike for previous channels our results are in better agreement with experiment
   than those of Sigaud \textit{et al}.\ which over estimate the cross sections to a greater extent and
   over a larger impact energy range. As in previous channels the under estimation of our cross sections
   at larger impact energies may be attributed to correlation effects, in particular, to antiscreening
   which  Sigaud \textit{et al}.\ seem to exaggerate in this channel.

   In to the outcome channels discussed above there are three additional processes. One, $\sigma_{10}$,
   has been left of as it involves no charge transfer. The other two, $\sigma_{00}$ and $\sigma_{30}$,
   involve all three electron coming to rest on either the target or the projectile. As was pointed out
   in the introduction these channels should not be considered due in part to the instability of the
   He\textsuperscript{-} ion and the fact that production cross sections for these configuration should
   be negligible. While modeling the initial and final states of the system as single Slater determinants
   accounts for Pauli exclusion precludes all three electrons from occupying the ground state there is
   nothing in the model to stop additional electron from capturing and remaining in excited states. The
   only recourse, short of implementing a model which contains at least some functional correlation, is
   to artificially redistribute the probability from $p_{00}$ and $p_{30}$ into other channels.

   Two options immediately present themselves. The first possibility is to feed the extra probability
   into the corresponding ionization channels. In other words, $p_{00}$ and $p_{30}$ would augment
   $p_{01}$ and $p_{11}$ respectively. This solution may provide a correction to some of the
   underestimation of $sigma_{01}$ around 40~keV/amu where $sigma_{30}$ peaks at approximately
   $0.14 \times 10^{-16}$ cm\textsuperscript{-16}, it would exacerbate the problems apparent as $E_P$
   approaches 10~keV/amu. More concerning, $\sigma_{00}$ is on the order of $i\sigma_{01}$. Incorporating
   this into $\sigma_{01}$ would effectively double its value for the lower impact energy range. This
   leaves one with the second option, to put the extra probabilities into $p_{10}$. In this case none
   of the results presented would be affected.

   One last point must be mentioned before closing this discussion. Above 500~keV/amu several of the cross
   section curves exhibit minor spurious structures. These are the result of numerical issues that, above
   this impact energy, limit the minimum possible impact parameter for which the calculations produce
   results from 0.1~a.u.\ below 500~keV/amu gradually to 0.8~a.u.\ at the 1000~kev/amu. In
   Eq.~\eqref{eq:cross} the integrandi, $b \, p(b)$, is approximated by a cubic spine which is, in turn,
   integrate to arrive at a cross section. The structure of the integrand means that so long as $p(0)$ is
   finite its value is irrelevant and we always know the integrand at $b = 0$. In the best case lower bound
   on the error of a cubic spline will scale to the third power in the largest step between
   knots~\cite{spline-err}. The step size factor in the error bound then increases from 0.001 to 0.512.
   It is this decrease in the accuracy of the interpolation which results in the minor structures above
   500~keV/amu. The merging of the n\textsc{etf} and p\textsc{etf} curves in the high energy limit for some
   channels.

\end{section}

\begin{section}{Conclusions \label{sec:conc}}

   In this work we have presented an investigation of the He\textsuperscript{+}-He collision within 
   time-dependent spin-density functional theory under the constraints of the exchange-only approximation.
   An accurate time-dependent exchange potential was determined through the application of the \textsc{kli}
   approximation. Total cross section results for all physical outcome channels were then offered in the
   approximate impact energy of 10-1000~keV/amu for two models. One in which electron translation factors
   were ignored and a second model where partial \textsc{etf}'s were used. The results of both models are
   in good agreement with experiment. Additionally, the current work is the only purely quantum mechanical
   approach which captures all outcome channels over such a wide range of impact energies.

   With out diminishing the results of this work it is necessary to highlight a few limitations and where
   the results may be improved in future iterations. First the limitation of the implementation of the 
   \textsc{kli} functional to systems of cylindrical symmetry is the impetus for both the 1s-only approximation
   as well as the need to consider both the n\textsc{etf} and p\textsc{etf} models. Future applications of
   the procedure laid out in this work would benefit greatly from a more robust implementation of the \textsc{kli}
   functional.

   Comparisons our results with the theoretical work of other groups points to the fact that the calculations
   would also gain much from the inclusion of correlation effects. Treating the x-only as a proof of concept
   there is noting, apart from the added complexity of the calculations, precluding the addition of dynamic
   correlation through the application of any number of ground-state correlation functionals in future.
   More difficult would be the inclusion of functional correlation effects. In order to move beyond the
   \textsc{iem} single Slater determinant description of outcome probabilities would necessitate the adaption
   of a model similar to that of Wilken and Bauer~cite{wb} beyond that of Ref.~\cite{p-he2p-he} to explicitly
   spin-polarized systems.

\end{section}

\begin{acknowledgments}

   This work was supported by the Natural Sciences and Engineering Research Council
   of Canada (\href{http://www.nserc-crsng.gc.ca/}{NSERC}) under grant number RGPIN-2014-03611.
   Additionally, this work was made possible by the facilities of the Shared Hierarchical Academic
   Research Computing Network (\href{www.sharcnet.ca}{SHARCNET}) and Compute/Calcul Canada. M.\ Baxter
   acknowledges the financial support provided by the Ontario Graduate and Queen Elizabeth II scholarships
   which are jointly funded by the province of Ontario and York University (Canada).

\end{acknowledgments}

\bibliography{hephe}

\end{document}

