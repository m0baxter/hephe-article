
\documentclass[aps, pra, reprint, groupedaddress, amsfonts, longbibliography,
               amsmath, amssymb, showpacs, nofootinbib]{revtex4-1}

\usepackage{microtype}
\usepackage{graphicx}
\usepackage{epstopdf}
\usepackage[utf8]{inputenc}
\usepackage[T1]{fontenc}
\usepackage[usenames,dvipsnames]{xcolor}
\usepackage{hyperref}
\usepackage{braket}

\begin{document}

\title{A time-dependent spin-density functional theory description of He\textsuperscript{+}-He
       collisions}
\author{Matthew Baxter}
\email[]{baxterma@yorku.ca}
\author{Tom Kirchner}
\email[]{tomk@yorku.ca}
\affiliation{Department of Physics and Astronomy, York University, Toronto, Ontario, Canada, M3J 1P3}
\author{Eberhard Engel}
\affiliation{Center for Scientific Computing, J.W. Goethe-Universit\"{a}t, D-60438 Frankfurt/Main,
             Germany}
\date{\today}

\begin{abstract}

   Theoretical total cross section results for all ionization/capture processes in the
   He\textsuperscript{+}-He collision system are presented in the approximate impact energy range
   10-1000~keV/amu. Calculations were performed within the framework of time-dependent spin-density
   functional theory. The Krieger-Li-Iafrate approximation was used to determine an accurate 
   exchange-correlation potential in the exchange-only limit. The results of two models, one where
   electron translation factors in the orbitals used to calculate the potential are ignored and another
   where partial electron translation factors are included, are compared with available experimental
   data as well as a selection of previous theoretical calculations.

\end{abstract}

\pacs{31.15.ee, 34.50.Fa, 34.10.+x, 34.70.+e (Maybe\dots)} %nnnote: probably change this

\maketitle

\begin{section}{Introduction \label{sec:intro}}

   The general collision system carries active electrons on both the target and projectile. The simplest
   generic collision system consists of a target with two electrons and a single electron on the
   projectile. A prototypical example from this class of problem is the He\textsuperscript{+}-He system.
   A popular alternative is to use an atomic hydrogen target (see for example~\cite{hEx1, hEx2}).
   However, in He\textsuperscript{+}-H collisions care must be taken when treating the distinct
   spin-singlet and triplet initial states~\cite{heph}.

   A variety of charge transfer processes are possible can occur as a result of these collisions. The
   processes can be broadly categorized into those that involve one active electron:
   %
   \begin{equation} \label{eq:tpi111}
      \mathrm{He}^+ + \mathrm{He} \rightarrow \mathrm{He}^+ + \mathrm{He}^+ + e^-
   \end{equation}
   %
   \begin{equation} \label{eq:tpi120}
      \mathrm{He}^+ + \mathrm{He} \rightarrow \mathrm{He} + \mathrm{He}^+
   \end{equation}
   %
   \begin{equation} \label{eq:tpi201}
      \mathrm{He}^+ + \mathrm{He} \rightarrow \mathrm{He}^{2+} + \mathrm{He} + e^-,
   \end{equation}
   %
   two active electrons:
   %
   \begin{equation} \label{eq:tpi012}
      \mathrm{He}^+ + \mathrm{He} \rightarrow \mathrm{He}^+ + \mathrm{He}^{2+} + 2e^-
   \end{equation}
   %
   \begin{equation} \label{eq:tpi102}
      \mathrm{He}^+ + \mathrm{He} \rightarrow \mathrm{He}^{2+} + \mathrm{He}^+ + 2e^-
   \end{equation}
   %
   \begin{equation} \label{eq:tpi021}
      \mathrm{He}^+ + \mathrm{He} \rightarrow \mathrm{He} + \mathrm{He}^{2+} + e^-,
   \end{equation}
   %
   and three active electrons:
   \begin{equation} \label{eq:tpi003}
      \mathrm{He}^+ + \mathrm{He} \rightarrow \mathrm{He}^{2+} + \mathrm{He}^{2+} + 3e^{-}.
   \end{equation}
   %
   Additionally there is the channel where no charges are transferred and the two channels that result
   in the production of He\textsuperscript{-}. From the point of view of experiment
   He\textsuperscript{-} production may be controlled for~\cite{metahe} and cross sections can even be
   made negligible~\cite{neghe-neg}.

   Total cross sections for the processes described in Eqs.~\eqref{eq:tpi111}-\eqref{eq:tpi003} have
   been calculated within a spin-density functional theoretic~\cite{td-spindep} framework, a
   generalization of time-dependent density functional theory~\cite{tddft, ullrich} to spin-dependent
   systems. The spin-polarized nature of the system accentuates the importance of exchange effects
   which in turn necessitates an accurate exchange potential. An exploration of a procedure for
   calculating such a potential comprises the bulk of this work.

   Experimental results for the various outcome channels over a large range of impact
   energies~\cite{BS58, HN78, HSE78, dCFdP88, DT-88, Dub-89, ASL91, FTFHLP-95, SSMSM-11} provide a
   useful benchmark for this problem. Results can also be compared with the theoretical work of other
   groups. These calculations employ a variety of methods including classical models such as the over
   the barrier model~\cite{CC-07}, Bohr–Lindhard model~\cite{DYC-08, DLZ-12}, and the classical
   trajectory Monte Carlo method~\cite{GMZ17}. Quantum mechanical calculations have been carried out
   using the local plasma approximation~\cite{MMA03}, the independent event model~\cite{SM-03}, as well
   as a plethora of perturbative calculations~\cite{Mancev96, BOC05, Mancev-07, MG-10, NTC11, GG-12b,
   GAG15}, the majority of which focus on single capture to the projectile [Eq.~\eqref{eq:tpi120}]. A
   calculation that would address all physical outcome channels is still outstanding. One of the primary
   objectives of the present work is to fill this void.

   We will begin our discussion with an overview of some relevant aspects of time-dependent density
   functional theory in Sec.~\ref{sec:tddft}. This will be followed by a description of the method used
   to calculate a time-dependent exchange potential in Sec.~\ref{sec:xpot}. This section also includes
   some details of the two-center basis generator method which was used to solve for the one-particle
   density. The theory section closes with a description of the final-state analysis for extracting
   the various outcome probabilities (Sec.~\ref{sec:probs}). The results of our calculations are
   presented in Sec.~\ref{sec:disc}. Finally conclusions drawn from these studies are offered in
   Sec.~\ref{sec:conc}.

   Atomic units ($\hbar = m_e = e = 1$) are used unless stated otherwise.

\end{section}

\begin{section}{Theory \label{sec:theory}}

   \begin{subsection}{TDDFT \label{sec:tddft}}

      A system of $N$ particles may be described by an $N$-particle wave function $\Psi(t)$ whose
      evolution is governed by the time-dependent Schr\"{o}dinger equation (\textsc{tdse})
      %
      \begin{equation} \label{eq:tdse}
         i \frac{\mathrm{d} \Psi(t)}{\mathrm{d}t} = \hat{H}(t) \Psi(t),
      \end{equation}
      %
      with a Hamiltonian $\hat{H}$ which may be written as
      %
      \begin{equation} \label{eq:ham}
         \hat{H}(t) = \hat{T} + \hat{V}_\mathrm{ee} + \hat{V}_\mathrm{ext}(t),
      \end{equation}
      %
      where $\hat{T}$ is the kinetic energy, $\hat{V}_\mathrm{ee}$ are the two particle interactions,
      and $\hat{V}_\mathrm{ext}$ is a time-dependent, one-particle interaction potential.

      The computationally intensive two-body term $\hat{V}_\mathrm{ee}$ makes direct solutions of the
      \textsc{tdse} difficult. Time-dependent density-functional theory~\cite{tddft, ullrich} (TDDFT)
      offers a solution to this problem. By making use of the mapping between the one-particle density
      %
      \begin{equation} \label{eq:dendef}
         n(\mathbf{r},t) = N \sum\limits_{\sigma} \int \mathrm{d}^3 r_2 \dots \mathrm{d}^3 r_N \,
                            \left| \Psi(\mathbf{x}_1,\dots, \mathbf{x}_N,t) \right|^2
      \end{equation}
      %
      and the external potential $\hat{V}_\mathrm{ext}$, where $x_i = (\mathbf{r}_i, \sigma_i)$ label
      the position and spin of the $i$\textsuperscript{th} particle, guaranteed by the Runge-Gross
      theorem~\cite{rg_theorem, td-spindep} the interacting system may be mapped onto a system of
      non-interacting particles.

      This so-called Kohn-Sham system consists of $N$ spin-orbitals $\varphi_{j \sigma}$ which evolve
      according to the time-dependent Kohn-Sham equation
      %
      \begin{equation} \label{eq:tdks}
         i \frac{\partial}{\partial t} \varphi_{j\sigma} = \left( -\frac{\Delta}{2} +
               v^\sigma_\mathrm{\textsc{ks}}[n_\uparrow, n_\downarrow](\mathbf{r},t)
               \right) \varphi_{j\sigma}(\mathbf{r},t),
      \end{equation}
      %
      such that
      %
      \begin{equation} \label{eq:ksden}
         n(\mathbf{r},t) = \sum\limits_\sigma \sum\limits_{j=1}^{N_\sigma}
                           \left| \varphi_{j\sigma}(\mathbf{r},t) \right|^2.
      \end{equation}
      
      In Eq.~\eqref{eq:tdks} $n_\sigma$, $\sigma \in \{ \uparrow, \downarrow \}$ are the spin-up/down
      one-particle densities given by
      %
      \begin{equation} \label{eq:spinden}
         n_\sigma = \sum\limits_{j=1}^{N_\sigma} \left| \varphi_{j\sigma} \right|^2,
      \end{equation}
      %
      $N_\sigma$ is the number of particles of a given spin projection $\sigma$, and
      $v^\sigma_\textsc{ks}$ is the Kohn-Sham potential. The potential may be decomposed into several
      simpler objects
      %
      \begin{equation} \label{eq:vks}
         v^\sigma_\mathrm{\textsc{ks}}[n_\uparrow, n_\downarrow] = v_\mathrm{ext} + v_\mathrm{H}[n]
                                                   + v^\sigma_\mathrm{xc}[n_\uparrow, n_\downarrow].
      \end{equation}
      %
      The first term in this expression is the external potential, which is essentially the potential
      $\hat{V}_\mathrm{ext}$ of Eq.~\eqref{eq:ham} renamed to match the single-particle notation. For
      the He\textsuperscript{+}-He collision system considered in this work $v_\mathrm{ext}$ may be
      written, making use of the semi-classical approximation, as
      %
      \begin{equation} \label{eq:hephe-ext}
         v_\mathrm{ext}(\mathbf{r},t) = -\frac{2}{r} 
         - \frac{2}{\left| \mathbf{r} - \mathbf{R}(t) \right|},
      \end{equation}
      %
      where $\mathbf{R}(t) = (b,0,V t)$ is the straight-line trajectory of the projectile with velocity
      $V$ and impact parameter (distance of closest approach) $b$.
      
      The next term in Eq.~\eqref{eq:vks} is the Hartree screening potential
      %
      \begin{equation} \label{eq:vh}
         v_\mathrm{H}(\mathbf{r},t) = \int \frac{n(\mathbf{r}^\prime, t)}
            {\left| \mathbf{r} - \mathbf{r}^\prime\right|} \, \mathrm{d}^3 r^\prime
      \end{equation}
      %
      which is an explicit functional of the full one-particle density. The last term is the
      exchange-correlation potential which encodes the complicated electron-electron interaction
      potential into the language of the non-interacting system. For convenience this is often further
      broken down into separate exchange and correlation potentials
      %
      \begin{equation} \label{eq:vxc}
         v^\sigma_\mathrm{xc} = v^\sigma_\mathrm{x} + v^\sigma_\mathrm{c}.
      \end{equation}

      Splitting $v_\mathrm{xc}$ into an exchange and correlation part facilitates the application of the
      x-only approximation where the correlation potential is taken to be zero ($v^\sigma_\mathrm{c} =
      0$). Such a model which ignores dynamic correlation is usually referred to as an independent
      electron model (\textsc{iem}). Within this approximation $v^\sigma_\mathrm{x}$ may be determined
      exactly via the optimized potential method~\cite{opm1, opm2, tdopm}. The complexity of the
      \textsc{opm} makes it prohibitively difficult to implement in general. As a secondary
      option one may instead make use of the Krieger-Li-Iafrate approximation~\cite{klieq, tdkli1,
      tdkli2} (\textsc{kli}). In many situations potentials generated using \textsc{kli} are numerically
      indistinguishable from those produced by the full \textsc{opm}~\cite{opm-rev}. The success of the
      \textsc{kli} is due to the fact that it preserves several properties of the exact potential. In
      particular \textsc{kli} maintains the correct asymptotics
      %
      \begin{equation} \label{eq:asymp}
         \lim\limits_{r \rightarrow \infty} v_\mathrm{x}^\sigma (\mathbf{r}) = -\frac{1}{r}.
      \end{equation}
   
   \end{subsection}

   \begin{subsection}{The exchange potential \label{sec:xpot}}

      The one-particle density was determined by solving Eq.~\eqref{eq:tdks} using the two-center basis
      generator method~\cite{tcbgm} (\textsc{tc-bgm}). As mentioned above this relies upon the
      specification of an exchange-correlation potential. While the correlation potential may be
      ignored, that is the x-only approximation may be used (with some understanding of the
      consequences), an accurate exchange-potential is essential for a precise description of the
      He\textsuperscript{+}-He collision system. The spin polarized nature of the system, which
      emphasizes exchange effects, makes this fact indisputable. To this end the \textsc{kli}
      approximation to the \textsc{opm} was employed in the calculation of $v^\sigma_\mathrm{x}$.

      The ground-state density functional theory (\textsc{dft}) scheme of~\cite{diamol} has been adapted
      to calculate a time-dependent exchange potential. At any instant of time, $t$, the
      He\textsuperscript{+}-He system may be regarded as a diatomic quasi-molecule with an internuclear
      distance $R_\mathrm{int}(t) = \sqrt{b^2 + Z(t)^2}$, where $b$ and $Z$ are the impact parameter and
      $z$ the position of the projectile as described below Eq.~\eqref{eq:hephe-ext}. If at each
      time-step of the \textsc{tc-bgm} the time-dependent Kohn-Sham orbitals,
      $\varphi_{\sigma j}(\mathbf{r},t)$ are fed into the \textsc{kli} functional (ignoring self
      consistency) an exchange-potential, $v^{\sigma}_\mathrm{x}[\varphi_{\sigma j};t]$, may be
      calculated at each $t$, effectively giving one the time-dependent exchange-potential
      $v^{\sigma}_\mathrm{x}[\varphi_{\sigma j}](t)$. The restriction of the \textsc{kli} scheme of
      Ref.~\cite{diamol} to systems of cylindrical symmetry complicates this processes since in general
      $\varphi_{j \sigma}$ will not exhibit any specific symmetry.
      
      In order to detail a solution to the symmetry problem a more thorough description of the
      \textsc{tc-bgm} is necessary. Within the \textsc{tc-bgm} the Kohn-Sham orbitals are represented in
      a non-orthogonal basis
      %
      \begin{equation} \label{eq:bgmexp}
         \varphi_{\sigma j}(\mathbf{r},t) = \sum\limits_{c \in \{P, T\}} \sum\limits_{k, L}
                               d_{c k L}^{\sigma j}(t) \tilde{\chi}^{L}_{c k}(\mathbf{r},t),
      \end{equation}
      %
      where
      %
      \begin{equation} \label{eq:etfbasis}
         \tilde{\chi}^{L}_{ck}(\mathbf{r},t) =
            \begin{cases}
               e^{i \mathbf{v}_T \cdot \mathbf{r}} {\chi}^{L}_{c k}(\mathbf{r},t) & c = T, \\[2ex]
               e^{i \mathbf{v}_P \cdot \mathbf{r}} {\chi}^{L}_{c k}(\mathbf{r},t) & c = P,
            \end{cases}
      \end{equation}
      %
      which are the basis functions with electron translation factors (\textsc{etf}) included. The basis
      functions themselves are given by
      %
      \begin{equation} \label{eq:bgmbasis}
         \chi^{L}_{ck} (\mathbf{r},t)
         = W_P( \mathbf{r},t, \epsilon_P)^L \chi^{0}_{ck} (\mathbf{r}),
      \end{equation}
      %
      with
      \begin{equation}
         W_P (\mathbf{r},t,\epsilon_P)
         = \frac{1 - e^{-\epsilon_P|\mathbf{r}_T - \mathbf{R}(t)|}}{|\mathbf{r}_T - \mathbf{R}(t)|},
      \end{equation}
      %
      where $\mathbf{r}_T$ represents the position vector with respect to the target center.
      
      In Eq.~\eqref{eq:bgmbasis} the functions $\chi^{0}_{ck}$ are the bound orbitals for the target
      helium atom ($c = T$) and the projectile helium ion ($c = P$). Additional states generated by a
      target potential operator are possible however in order to keep the number of states in the basis
      to a minimum and simplify the description only the pseudostates generated with $W_P$ are included.
      This simplification has proven sufficient in the past~\cite{bgm-rev}. The remaining regularizer is
      set to $\epsilon_P = 1$. The complete basis set used may be described in terms of the maximum $L$
      value included for each bound sub-shell, 1$s$-4$f$, in the basis. On each center these were taken
      to be (0 0 1 1 1 2 2 2 2 2 2 3 3 3 3 3 3 3 3 3), for a total of 124 basis states.

      It is clear from the above description that only the basis states corresponding to s-type orbitals
      will make cylindrically symmetric contributions to the \textsc{ks}-orbitals. The simplest solution
      is to only feed the 1s contributions into the \textsc{kli} functional. While higher s-states are
      also admissible they will no longer represent the most occupied subshell meaning their inclusion
      will do little to improve the accuracy of the description of a given orbital. Leaving out
      \textsc{etf}s for the time being the orbitals will take the explicit form
      %
      \begin{equation} \label{eq:1sonly}
         \varphi_{\sigma j}^{1s}(\mathbf{r},t) = a^{\sigma j}_T(t) \chi^{0}_{T1}(\mathbf{r},t)
                                               + a^{\sigma j}_P(t) \chi^{0}_{P1}(\mathbf{r},t).
      \end{equation}
      %
      The coefficients are the result of projecting the \textsc{ks}-orbitals onto the two-dimensional
      subspace spanned by the target and projectile 1s-states
      %
      \begin{equation} \label{eq:proj}
         \ket{\varphi^{1s}_{\sigma j}} = \hat{P} \ket{\varphi_{\sigma j}}
                                       = \sum\limits_{c_1, c_2 \in \{T,P\}} \tilde{S}^{-1}_{c_1 c_2}
                                                      \ket{\chi^{0}_{c_1 1}}
                                                      \braket{\chi^{0}_{c_2 1}| \varphi_{\sigma j}}
      \end{equation}
      %
      with $\tilde{S}^{-1}_{c_1 c_2}$ the inverse of the overlap matrix
      %
      \begin{equation} \label{eq:overlap2d}
         \tilde{S}_{c_1 c_2} = \braket{ \chi^{0}_{c_1 1} | \chi^{0}_{c_2 1} }.
      \end{equation}
      %
      The coefficients are then determined to be
      %
      \begin{equation} \label{eq:coef}
         a^{\sigma j}_c = \sum\limits_{c_1, c_2 \in \{T,P\}} \sum\limits_{k = 1}^K
                          \sum\limits_{l = 0}^L \tilde{S}^{-1}_{c c_1} S^{c_2 \, k \, l}_{c \, 1 \, 0}
                             d^{\sigma j}_{c_2 k l},
      \end{equation}
      %
      with
      %
      \begin{equation} \label{eq:overlap}
         S_{c_1 \, k_1 \, l_1 }^{c_2 \, k_2 \, l_2 } =
            \braket{ \chi^{l_1}_{c_1 k_1} | \chi^{l_2}_{c_2 k_2} }
      \end{equation}
      %
      the full overlap matrix.

      Returning to the question of the \textsc{etf}s, working in the rotating center of mass frame in
      which the $z$-direction points along the internuclear axis the \textsc{etf}s become
      %
      \begin{subequations} \label{eq:etf}
         \begin{equation} \label{eq:etfT}
            e^{i \mathbf{v}_T \cdot \mathbf{r}} =
             e^{\frac{i V_\mathrm{rel}}{2} (x \sin \theta - z \cos \theta)},
         \end{equation}
         %
         \begin{equation} \label{eq:etfP}
            e^{i \mathbf{v}_P \cdot \mathbf{r}} =
             e^{\frac{i V_\mathrm{rel}}{2} (z \cos \theta - x \sin \theta)},
         \end{equation}
      \end{subequations}
      %
      where $V_\mathrm{rel} = V$ is the relative velocity of the centers and $\theta = \arctan b/z$
      [where $V$ is the velocity appearing below Eq.~\eqref{eq:hephe-ext}]. If we now introduce a
      two-centered coordinate system, placing the foci at the two nuclear centers
      %
      \begin{equation} \label{eq:psc}
            \left\{
            \begin{array}{l}
               x = \frac{|R|}{2} \sqrt{(\xi^2 - 1)(1 - \eta^2)} \sin \phi, \\ [2ex]
               y = \frac{|R|}{2} \sqrt{(\xi^2 - 1)(1 - \eta^2)} \cos \phi, \\ [2ex]
               z = \frac{|R|}{2} \xi \eta,
            \end{array}
            \right.
      \end{equation}
      %
      it becomes clear that the portion of the \textsc{etf}s containing $x$ violates the desired
      cylindrical symmetry.

      Two obvious solutions present themselves. First, one may simply ignore the \textsc{etf}s
      completely. This would amount to passing the orbitals described by Eq.~\eqref{eq:1sonly} into
      the \textsc{kli} functional. Alternatively the symmetry breaking portions of the \textsc{etf}s
      may be dropped. In this case the full 1s-only \textsc{ks}-orbital becomes
      %
      \begin{equation} \label{eq:1sonlyetf}
         \tilde{\varphi}_{\sigma j}^{1s} (\mathbf{r},t) =
                       a^{\sigma j}_T (t)  e^{- \frac{i v_\mathrm{rel} z \cos \theta}{2}}
                                           \chi^{0}_{T1} (\mathbf{r},t)
                     + a^{\sigma j}_P (t)  e^{  \frac{i v_\mathrm{rel} z \cos \theta}{2}}
                                           \chi^{0}_{P1} (\mathbf{r},t).
      \end{equation}
      %
      This will offer at least some of the correction provided by the full \textsc{etf}. Unfortunately,
      as the electronic coordinate $Z(t)$ approaches zero the partial \textsc{etf} will tend to one,
      this being the closest approach where $\theta = \frac{\pi}{2}$, meaning that when the target and
      projectile are at their closest, the most active region of the collision, no \textsc{etf} will be
      present. 
      
      Regardless of which option is chosen it is important that $v_\mathrm{H}$ be determined with the
      same set of orbitals used in the calculation of $v_\mathrm{x}^\sigma$, preserving the precise
      cancellation of the self interaction present in the Hartree potential.

   \end{subsection}

   \begin{subsection}{Final-state analysis \label{sec:probs}}

      Of interest in any scattering problem is the probability of finding the system in some final state.
      If we represent the state being considered as $\ket{f_1 \, f_2 \, f_3}$ and the initial state of
      the system propagated to some final time $t_f$ by $\ket{\varphi_{ \uparrow 1} \,
      \varphi_{\uparrow 2} \, \varphi_{\downarrow 1} (t_f)}$ the exclusive probability to find the
      system in the given final state at time $t_f$ will be given by
      %
      \begin{equation} \label{eq:finalProb}
         P_{f_1 f_2 f_3}(t_f) = \left| \braket{ f_1 \, f_2 \, f_3 | \varphi_{ \uparrow 1} \,
                                       \varphi_{\uparrow 2} \, \varphi_{\downarrow 1} (t_f) } \right|^2.
      \end{equation}
      %
      If one assumes that both the propagated and final states can be represented as a single Slater
      determinant then the probability in question becomes
      %
      \begin{equation} \label{eq:detProb}
         P_{f_1 f_2 f_3}(t_f) = \det \left[ \gamma_{f f^\prime}(t_f) \right],
      \end{equation}
      %
      where $\gamma_{f f^\prime}$ is the one-particle density matrix
      %
      \begin{equation} \label{eq:denmat}
         \gamma_{f f^\prime}(t_f) = \sum\limits_{\sigma} \sum\limits_{j = 1}^{N_\sigma}
                               \braket{f | \varphi_{\sigma j}(t_f)}
                               \braket{\varphi_{\sigma j}(t_f) | f^\prime},
      \end{equation}
      %
      with the transition amplitudes $\braket{f | \varphi_{\sigma j}(t_f)} =
      \braket{\chi^l_{c k} | \varphi_{\sigma j}(t_f)}$ (for some $k$, $l$, and $c$) readily calculable
      from the dynamics. A model of this type which ignores the functional correlations~\cite{p-he2p-he}
      is consistent with an \textsc{iem} description.

      Alternatively, one could consider the probability to explicitly measure the states of some subset
      of the total number of particles. These so called inclusive probabilities can be expressed in
      terms of determinants of submatrices of the density matrix~\cite{inc-prob}.
 
      In the current problem we are interested in those probabilities which correspond to the outcome
      channels of Eqs.~\eqref{eq:tpi111}-\eqref{eq:tpi003}. In such configurations we find $k$ particles
      on the projectile, $l$ particles in the continuum, and $3 - k - l$ on the target ($0\leq k \leq 3$
      and $0 \leq l \leq 3 - k$). The probabilities $p_{kl}$ may be calculated in terms of sums of
      inclusive probabilities to find a given number of particles in the bound states of the target and
      projectile by applying the machinery of~\cite{inc-prob} (see for example~\cite{incEx, mitsuko12,
      gerald15}). With the probabilities in hand the corresponding total cross section for each channel
      may then be calculated from (if we ignore $\sigma_{10}$ which includes the elastic channel and
      must be treated with more care)
      %
      \begin{equation} \label{eq:cross}
         \sigma_{kl} = 2 \pi \int_0^\infty b \, p_{kl}(b) \, \mathrm{d}b.
      \end{equation}

   \end{subsection}

\end{section}

\begin{section}{Discussion \label{sec:disc}}

   \begin{figure}[t]
      \centering
      \includegraphics[width = \linewidth]{./images/hephe-pb-40.eps}
      \caption{Single particle probabilities for each spin orbital to transfer between
               centers (thick lines) and to ionize (thin lines) in the p\textsc{etf} model.
               \label{fig:pbPlot}}
   \end{figure}

   \begin{figure}[t]
      \centering
      \includegraphics[width = \linewidth]{./images/crossPlots/HepHe-111.eps}
      \caption{Total cross section for single ionization of the target.
               Theoretical: p\textsc{etf} (solid line), n\textsc{etf} (dotted line), and
                            \textsc{cdw-eis} of Miraglia and Gravielle~\cite{MG-10} (dashed line).
               Experimental: diamonds~\cite{Dub-89}, circles~\cite{FTFHLP-95}, and squares~\cite{DT-88}.
               \label{fig:cs111}}
   \end{figure}

   \begin{figure}[t]
      \centering
      \includegraphics[width = \linewidth]{./images/crossPlots/HepHe-201.eps}
      \caption{Total cross section for single ionization of the projectile.
               Theoretical: p\textsc{etf} (solid line),n\textsc{etf} (dotted line), and
                            \textsc{ievm} of Sigaud \textit{et al}.~\cite{SM-03} (dashed line).
               Experimental: diamonds~\cite{Dub-89}. \label{fig:cs201}}
   \end{figure}

   \begin{figure}[t]
      \centering
      \includegraphics[width = \linewidth]{./images/crossPlots/HepHe-120.eps}
      \caption{Total cross section for single capture to the projectile.
               Theoretical: p\textsc{etf} (solid line), n\textsc{etf} (dotted line), and
                            \textsc{cdbw-4b} (post form) of Ghanbari-Adivi \textit{et al}.~\cite{GAG15}
                            (dashed line).
               Experimental: diamonds~\cite{Dub-89} and circles~\cite{FTFHLP-95}. \label{fig:cs120}}
   \end{figure}

   \begin{figure}[t]
      \centering
      \includegraphics[width = \linewidth]{./images/crossPlots/HepHe-012.eps}
      \caption{Total cross section for double ionization of the target.
               Theoretical: p\textsc{etf} (solid line) and n\textsc{etf} (dotted line).
               Experimental: diamonds~\cite{Dub-89}, circles~\cite{FTFHLP-95}, and squares~\cite{DT-88}.
               \label{fig:cs012}}
   \end{figure}

   \begin{figure}[t]
      \centering
      \includegraphics[width = \linewidth]{./images/crossPlots/HepHe-021.eps}
      \caption{Total cross section for transfer ionization of the target.
               Theoretical: p\textsc{etf} (solid line) and n\textsc{etf} (dotted line).
               Experimental: diamonds~\cite{Dub-89} and circles~\cite{FTFHLP-95}. \label{fig:cs021}}
   \end{figure}
   \begin{figure}[t]
      \centering
      \includegraphics[width = \linewidth]{./images/crossPlots/HepHe-102.eps}
      \caption{Total cross section for simultaneous single ionization of the target and projectile.
               Theoretical: p\textsc{etf} (solid line), n\textsc{etf} (dotted line), and
                            \textsc{ievm} of Sigaud \textit{et al}.~\cite{SM-03} (dashed line).
               Experimental: diamonds~\cite{Dub-89} and crosses~\cite{SSMSM-11}. \label{fig:cs102}}
   \end{figure}

   \begin{figure}[t]
      \centering
      \includegraphics[width = \linewidth]{./images/crossPlots/HepHe-003.eps}
      \caption{Total cross section simultaneous double target and single projectile ionization.
               Theoretical: p\textsc{etf} (solid line), n\textsc{etf} (dotted line), and
               \textsc{ievm} of Sigaud \textit{et al}.~\cite{SM-03} (dashed line).
               Experimental: diamonds~\cite{Dub-89} and crosses~\cite{SSMSM-11}. \label{fig:cs003}}
   \end{figure}

   In what follows all results obtained by propagating the full \textsc{ks} orbitals in a potential
   generated from the 1s-only orbitals of Eq.~\eqref{eq:1sonly} that includes no electron translation
   factors will be designated by n\textsc{etf}. Those obtained by an application of the same processes
   using the 1s-only orbitals, with partial \textsc{etf}s, of Eq.~\eqref{eq:1sonlyetf} will be referred
   to as p\textsc{etf}.

   Before discussing the total cross section results we will present some of the lower level features of
   the calculations. We will consider the single-particle probabilities for each orbital to ionize and
   to switch the nuclear center to which it is bound. These probabilities can be calculated from the
   transition amplitudes. As an example, if $\varphi_{\uparrow 1}$ begins initially on the target then
   the single-particle probability to ionize this electron can be written
   %
   \begin{equation} \label{eq:ionprob}
      p(\mathrm{He}(\uparrow_1) \rightarrow I ) =
         \sum\limits_{c \in \{T,P\}} \sum\limits_{k=1}^K \sum\limits_{l = 1}^L
         \left| \braket{ \chi^{l}_{c k} | \varphi_{\uparrow 1}(t_f)} \right|^2
   \end{equation}
   %
   and the single-particle probability to transfer to the projectile may be written
   %
   \begin{equation} \label{eq:transprob}
      p(\mathrm{He}(\uparrow_1) \rightarrow \mathrm{He}^+) =
         \sum\limits_{k=1}^K \left| \braket{ \chi^{0}_{P k} | \varphi_{\uparrow 1}(t_f)} \right|^2.
   \end{equation}
   %
   These probabilities are presented for the p\textsc{etf} model for an impact energy ($E_P =
   \frac{1}{2} m_\textrm{He} V^2$) of 40~keV/amu in Fig.~\ref{fig:pbPlot}. At this impact energy
   capture is the the dominant process. As one would expect the probability to ionize the more tightly
   bound He\textsuperscript{+} electron is less consistently less then for either of the He electrons.
   Also of note is the obvious difference between the two He electrons, clear evidence that the
   implementation of a spin-dependent potential was not in vein.

   Total cross sections for the processes described in
   Eqs.~\eqref{eq:tpi111}-\eqref{eq:tpi003} are presented in Figs.~\ref{fig:cs111}-\ref{fig:cs003}.
   Where available the results of the current work are compared with calculations of other groups. It
   should be noted that only those calculations that describe the system quantum mechanically were
   considered, that is to say works that employ approaches such as the classical trajectory Monte Carlo
   method are not included.

   We begin the discussion by considering the single target ionization process of Eq.~\eqref{eq:tpi111}.
   The results for this channel, $\sigma_{11}$ are presented in Fig.~\ref{fig:cs111}. Both n\textsc{etf}
   and p\textsc{etf} are in good agreement with experiment throughout the full range of impact energies.
   As impact energy increases a slight gap opens between the two. This trend is, as one would expect,
   due to the fact the \textsc{etf}s should become more relevant as the relative velocity between
   target and projectile increases. The slightly lower values for the p\textsc{etf} version above
   200~keV/amu make it a better fit to experiment. The underestimation of both curves below the peak
   correspond almost exactly with the region where $\sigma_{20}$ results begin to rise above the
   experimental results (see Fig.~\ref{fig:cs120}).

   Also displayed in Fig.~\ref{fig:cs111} are the continuum-distorted-wave eikonal-initial-state
   approximation (\textsc{cdw-eis}) results of Miraglia and Gravielle~\cite{MG-10}. These results seem
   to compliment the results of the present work through the majority of the impact energy range. One
   notable exception to this is the slightly higher cross section maximum however. As there is a fairly
   large spread in the experimental data around this region it is difficult to say which is more
   accurate. The results of Miraglia and Gravielle also begin to diverge as one approaches lower
   impact energies. This feature is likely due in large part the perturbative nature of \textsc{cdw-eis}
   which becomes less reliable as one decreases the impact energy.

   Next, we consider the results for $\sigma_{01}$ [Eq.~\eqref{eq:tpi201}] shown in
   Fig.~\ref{fig:cs201}. As with the previous channel both n\textsc{etf} and p\textsc{etf} results are
   in reasonable agreement with the experiment where it is available. Also continuing the trend seen in
   the $\sigma_{11}$ results both models begin essentially equal at low impact energies and diverge as
   $E_P$ increases. Both models begin to over estimate the data above the peak around 200~keV/amu. Once
   again the slightly lower p\textsc{etf} results are in better agreement with experiment. The slight
   unphysical structures in the curves below 40~keV/amu seem to correspond with the peaks of the
   $\sigma_{00}$ channel (not pictured here). This issue will be discussed in greater detail below.

   These calculations have been compared with the independent event model (\textsc{ievm}) results of
   Sigaud \textit{et al}.~\cite{SM-03}. While they do not directly report $\sigma_{01}$ they do present
   $\sigma_{02}$, $\sigma_{03}$, and what they call total electron loss (we will denote this by
   $\sigma_\mathrm{total}$). Using the relation
   %
   \begin{equation} \label{eq:total}
      \sigma_\mathrm{total} = \sigma_{01} + \sigma_{02} + \sigma_{03}
   \end{equation}
   %
   one can easily determine $\sigma_{01}$ from their disclosed results. Their values seem to be in much
   better agreement with experiment in the high energy range than either n\textsc{etf} or p\textsc{etf}
   models. This can perhaps be explained by the presence of correlation in the \textsc{ievm}
   calculations.
   %MOVE
   In particular,    %MOVE

   For the results of single electron capture to the projectile, the process of Eq.~\eqref{eq:tpi120}
   depicted in Fig.~\ref{fig:cs120}, both n\textsc{etf} and p\textsc{etf} models are essential
   identical. This is what one would hope for as they are in good agreement with experiment in the
   entire range of impact energies. A possible explanation of the slight discrepancy between theory and
   experiment in the 50-150~keV/amu interval is offered by a comparison with the four-body Coulomb–Born
   distorted wave approximation (\textsc{cdbw-4b}) results of Ghanbari-Adivi
   \textit{et al}.~\cite{GAG15}. The correlation effects included in this model may point to the slight
   rise in cross section being related to either the fact that we have employed an \textsc{iem}
   approximation or a failure of the partial \textsc{etf}.

   The later explanation may provide a more satisfying solution to this problem. One would expect that
   capture processes should be dominate by the contributions of slow and close collisions. The regions
   where the n/p\textsc{etf} models start to diverge from experiment is approximately the region where
   both models begin to diverge in other channels [see for example $\sigma_{12}$ in
   Fig.~\ref{fig:cs012}], that is the lowest energies where \textsc{etf}s are important.
   Additionally they begin to agree with experiment once the cross sections begin to rapidly approach
   zero, for fast collisions. This would seem to be an indication that correct \textsc{etf}s are
   important for capture processes (a fact that should be at least intuitively obvious).

   A few words should be spent addressing the choice for the theoretical calculation compared against.
   Unlike other channels there exists a relatively large number of works to select from that fit the
   criteria listed above. As the majority of these belong to a family of related perturbative
   models~\cite{Mancev96, BOC05, Mancev-07, MG-10, NTC11, GG-12b, GAG15} the latest, that of
   Ghanbari-Adivi \textit{et al}., was chosen. A comparison of the work of Ghanbari-Adivi
   \textit{et al}.\ with several earlier perturbative calculations can be found in Ref.~\cite{GAG15}.

   With the single-electron processes taken care of double target ionization, Eq.~\eqref{eq:tpi012},
   the first of the two-electron processes will be considered next. The results for this channel are
   presented in Fig.~\ref{fig:cs012}. As with previous channels both n\textsc{etf} and p\textsc{etf}
   results appear to be very similar with a slight edge going to the p\textsc{etf} model's marginally
   lower results above 100~keV/amu. Both models seem to shift the peak in the cross section to higher
   impact energy than experiment would suggest is correct. As one would expect from an \textsc{iem}
   the two models exaggerate double ionization, see for example Refs~\cite{pbarhe-rev, p-he2p-he}. As
   there are no previous works fitting the conditions for inclusion listed earlier little else can be
   concluded from about the results of the present work.

   Another channel where the literature lacks a proper touchstone is that of transfer ionization\
   [Eq.~\eqref{eq:tpi021}]. The trends for $\sigma_{21}$ are very similar to those of $\sigma_{12}$.
   As with the previously discussed process both models are above experiment and shift the experimental
   peak to a higher impact energy. The only significant difference is that this is one of the few
   channels where the n\textsc{etf} model is in slightly better agreement with experiment than the
   p\textsc{etf}. The flattening of the curves below 20~keV/amu is an artifact of the \textsc{tc-bgm}
   becoming less reliable at the lowest impact energies.

   The last two-electron process is simultaneous single ionization of the target and projectile,
   Eq.~\eqref{eq:tpi102}. The results for our n\textsc{etf} and p\textsc{etf} models are presented in
   Fig.~\ref{fig:cs102}. These results both follow the trend of the data quite closely, arguably
   matching the position of the peak in the experimental cross sections. This channel is the second of
   two where the n\textsc{etf} model has a slight egde over the results of the p\textsc{etf}.
   Unfortunately they fall below experiment for the majority of the impact energy range shown.

   A comparison with the \textsc{ievm} of Sigaud \textit{et al}.\ explains this fact. Sigaud
   \textit{et al}.\ claim to capture the effects of antiscreening, the direct interaction between target
   and projectile electrons, which becomes increasingly important for projectile ionization processes at
   larger impact energies. As the results of the current work are those of an independent
   electron model (\textsc{iem}) they make no effort to capture any correlation effects. Sigaud 
   \textit{et al}.'s efforts to capture antiscreening see their results fall within experiment for their
   entire extent. Encouragingly, if one were to extend the curve of Sigaud \textit{et al}.\ it would
   seem to overlap with the results of the present work lending credence to the curve in the region
   below the cross section peak, where antiscreening cannot contribute.

   Finally we will consider the sole three electron process, simultaneous double target and single
   projectile ionization [Eq.~\eqref{eq:tpi003}]. The results, presented in Fig.~\ref{fig:cs003}, again
   follow the general trends found in the previously discussed channels; overestimation of the cross
   section peak and a slightly better showing for the p\textsc{etf} model over the results of the
   n\textsc{etf} model. Unlike for previous channels our results are in better agreement with experiment
   than those of Sigaud \textit{et al}.\ which over estimate the cross sections to a greater extent and
   over a larger impact energy range. As in previous channels the under estimation of our cross sections
   at larger impact energies may be attributed to correlation effects, in particular, to antiscreening
   which  Sigaud \textit{et al}.\ seem to exaggerate in this channel.

   In addition to the outcome channels discussed above there are three additional processes. One,
   $\sigma_{10}$, has been left out as it involves no charge transfer. The other two, $\sigma_{00}$ and
   $\sigma_{30}$, involve all three electrons bound to either the target or the projectile. As was
   pointed out in the introduction these channels should not be considered due in part to the
   instability of the He\textsuperscript{-} ion and the fact that production cross sections for these
   configuration should be negligible. While modeling the initial and final states of the system as
   single Slater determinants accounts for Pauli exclusion which precludes all three electrons from
   occupying the ground state there is nothing in the model to stop additional electrons from capturing
   and remaining in excited states. The only recourse, short of implementing a model which contains at
   least some functional correlation, is to artificially redistribute the probability from $p_{00}$ and
   $p_{30}$ into other channels.

   Two options immediately present themselves. The first possibility is to feed the extra probability
   into the corresponding ionization channels. In other words, $p_{00}$ and $p_{30}$ would augment
   $p_{01}$ and $p_{21}$ respectively. With the peak in $\sigma_{30}$ approximately matching that of
   $\sigma_{21}$ in both position and magnitude this solution would lead to a doubling of the over
   estimation present in the $\sigma_{21}$ channel. A similar issue would arise in the lower impact
   energy range of the $\sigma_{01}$ curve. This leaves one with the second option, to put the extra
   probability from $p_{00}$ into $p_{10}$ and feed $p_{30}$ into $p_{20}$. The only effect this could
   have on the presented results would be to increase $\sigma_{20}$ however, as $\sigma_{30}$ is at
   worst an order of magnitude less than $\sigma_{20}$ it would provide only a small shift in the curve
   displayed in Fig.~\ref{fig:cs120}.

   One last point must be mentioned before closing this discussion. Above 500~keV/amu several of the
   cross section curves exhibit minor spurious structures. These are the result of numerical issues
   that, above this impact energy, limit the minimum possible impact parameter for which the
   calculations produce results from 0.1~a.u.\ below 500~keV/amu gradually to 0.8~a.u.\ at 1000~keV/amu.
   In Eq.~\eqref{eq:cross} the integrand, $b \, p(b)$, is approximated by a cubic spline which is, in
   turn, integrated to arrive at a cross section. The structure of the integrand means that so long as
   $p(0)$ is finite its value is irrelevant and we always know the integrand at $b = 0$. In the best
   possible scenario the lower bound on the error of a cubic spline will scale to the third power in the
   largest step between knots~\cite{spline-err}. The step size factor in the error bound then increases
   from 0.001 to 0.512, an increase by an approximate factor of 500. It is this decrease in the accuracy
   of the interpolation which results in the minor structures above 500~keV/amu. The presence of these
   unphysical structures are the reason for the lack of a data point at 1000~keV/amu in
   Fig.~\ref{fig:cs201}.

\end{section}

\begin{section}{Conclusions \label{sec:conc}}

   In this work we have presented an investigation of the He\textsuperscript{+}-He collision system
   within  time-dependent spin-density functional theory under the constraints of the exchange-only
   approximation. An accurate time-dependent exchange potential was determined through the application
   of the \textsc{kli} approximation. Total cross section results for all physical outcome channels were
   then offered in the approximate impact energy of range 10-1000~keV/amu for two models. One in which
   electron translation factors were ignored and a second model where partial \textsc{etf}s were used.
   The results of both models are in overall good agreement with experiment. Additionally, the current
   work is the only purely quantum mechanical approach which captures all outcome channels over such a
   wide range of impact energies.

   Without diminishing the results of this work it is necessary to highlight a few limitations and where
   the results may be improved in future iterations. First, the restriction of the implementation of the 
   \textsc{kli} functional to systems of cylindrical symmetry is the impetus for both the 1s-only
   approximation as well as the need to consider both the n\textsc{etf} and p\textsc{etf} models. Future
   applications of the procedure laid out in this work would benefit greatly from a more fully
   three-dimensional implementation of the \textsc{kli} functional that makes no symmetry assumptions.

   Comparisons of our results with the theoretical works of other groups points to the fact that the
   calculations would also gain from the inclusion of correlation effects. Treating this x-only version
   as a proof of concept there is nothing, apart from the added complexity of the calculations,
   precluding the addition of dynamic correlation through the application of any number of ground-state
   correlation functionals in the future. It should be noted that such a model would still not offer a
   complete description of time-dependent correlation, it would, for example, lack memory
   effects~\cite{tddft}. An added difficulty would be the inclusion of functional correlation effects.
   In order to move beyond the \textsc{iem} single Slater determinant description of outcome
   probabilities would have to adapt a model similar to that of Wilken and Bauer~cite{wb} beyond that of
   Ref.~\cite{p-he2p-he} to explicitly spin-polarized systems.

\end{section}

\begin{acknowledgments}

   This work was supported by the Natural Sciences and Engineering Research Council
   of Canada (\href{http://www.nserc-crsng.gc.ca/}{NSERC}) under grant number RGPIN-2014-03611.
   Additionally, this work was made possible by the facilities of the Shared Hierarchical Academic
   Research Computing Network (\href{www.sharcnet.ca}{SHARCNET}) and Compute/Calcul Canada. M.\ Baxter
   acknowledges the financial support provided by the Ontario Graduate and Queen Elizabeth \textsc{II}
   scholarships which are jointly funded by the province of Ontario and York University (Canada).

\end{acknowledgments}

\bibliography{hephe}

\end{document}

