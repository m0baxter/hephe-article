
\documentclass[aps, pra, reprint, groupedaddress, amsfonts,
               amsmath, amssymb, showpacs, nofootinbib]{revtex4-1}

\usepackage{microtype}
\usepackage{graphicx}
\usepackage{epstopdf}
\usepackage[utf8]{inputenc}
\usepackage[T1]{fontenc}
\usepackage[usenames,dvipsnames]{xcolor}
\usepackage{hyperref}
\usepackage{braket}

\begin{document}

\title{Ceci n'est pas un titre}
\author{Matthew Baxter}
\email[]{baxterma@yorku.ca}
\author{Tom Kirchner}
\email[]{tomk@yorku.ca}
\affiliation{Department of Physics and Astronomy, York University, Toronto, Ontario, Canada, M3J 1P3}
\author{Eberhard Engel}
\affiliation{Center for Scientific Computing, J.W. Goethe-Universit\"{a}t, D-60438 Frankfurt/Main,
             Germany}
\date{\today}

\begin{abstract}
   !`Empty!
\end{abstract}

\pacs{31.15.ee, 34.50.Fa, 34.10.+x, 34.70.+e (Maybe\dots)} %nnnote: probably change this

\maketitle

\begin{section}{Introduction \label{sec:intro}}
\end{section}

\begin{section}{Theory \label{sec:theory}}

   \begin{subsection}{TDDFT \label{sec:tddft}}

      A system of $N$ particles may be described by an $N$-particle wave function $\Psi(t)$ whose
      evolution is governed by the time-dependent Shr\"{o}dinger equation (\textsc{tdse})
      %
      \begin{equation} \label{eq:tdse}
         i \frac{\mathrm{d} \Psi(t)}{\mathrm{d}t} = \hat{H}(t),
      \end{equation}
      %
      with a Hamiltonian $\hat{H}$ which may be written as
      %
      \begin{equation} \label{eq:ham}
         \hat{H}(t) = \hat{T} + \hat{V}_\mathrm{ee} + \hat{V}_\mathrm{ext}(t),
      \end{equation}
      %
      where $\hat{T}$ is the kinetic energy, $\hat{V}_\mathrm{ee}$ are the two particle interactions,
      and $\hat{V}_\mathrm{ext}$ is a time-dependent, one-particle interaction potential.

      The computationally intensive two-body term $\hat{V}_\mathrm{ee}$ makes direct solutions of the
      \textsc{tdse} difficult. Time-dependent density-functional theory~\cite{tddft, ullrich} (TDDFT)
      offers a solution to this problem. By making use of the mapping between the one-particle density
      %
      \begin{equation} \label{eq:dendef}
         n(\mathbf{r},t) = N \sum\limits_{sigma} \int \mathrm{d}^3 r_2 \dots \mathrm{d}^3 r_N \,
                            \left| \Psi(\mathbf{x}_i,t) \right|^2
      \end{equation}
      %
      and the external potential $\hat{V}_\mathrm{ext}$, where $x_i = (\mathbf{r}_i, \sigma_i)$ label
      the position and spin of the $i$\textsuperscript{th} particle, guaranteed by the Runge-Gross
      theorem~\cite{rg_theorem, td-spindep} the interacting system may be mapped onto as system of
      non-interacting particles.

      This so-called Kohn-Sham system consists of $N$ spin-orbitals $\varphi_{j \sigma}$ which evolve
      according to the time-dependent Kohn-Sham equation
      %
      \begin{equation} \label{eq:tdks}
         i \frac{\partial}{\partial t} \varphi_{j\sigma} = \left( -\frac{\Delta}{2} +
               v^\sigma_\mathrm{\textsc{ks}}[n_\uparrow, n_\downarrow](\mathbf{r},t)
               \right) \varphi_{j\sigma}(\mathbf{r},t),
      \end{equation}
      %
      such that
      %
      \begin{equation} \label{eq:ksden}
         n(\mathbf{r},t) = \sum\limits_\sigma \sum\limits_{j=1}^{N_\sigma}
                           \left| \varphi_{j\sigma} \right|^2.
      \end{equation}
      
      In Eq.~\eqref{eq:tdks} $n_\sigma$ are the spin-up/down one-particle densities given by
      %
      \begin{equation} \label{eq:spinden}
         n_\sigma = \sum\limits_{j=1}^{N_\sigma} \left| \varphi_{j\sigma} \right|^2
      \end{equation}
      %
      and $v^\sigma_\textsc{ks}$ is the Kohn-Sham potential. The potential may be separated into several
      simpler objects
      %
      \begin{equation} \label{eq:vks}
         v^\sigma_\mathrm{\textsc{ks}}[n_\uparrow, n_\downarrow] = v_\mathrm{ext} + v_\mathrm{H}[n]
                                                   + v^\sigma_\mathrm{xc}[n_\uparrow, n_\downarrow].
      \end{equation}
      %
      The first term in this expression is the external potential, which is essentially the same as the
      potential $\hat{V}_\mathrm{ext}$ of Eqs.~\eqref{eq:ham}. For the He\textsuperscript{+}-He
      collision system considered in this work it may be written, making use of the semi-classical
      approximation,
      %
      \begin{equation} \label{eq:hephe-ext}
         v_\mathrm{ext}(\mathbf{r},t) = -\frac{2}{r} 
         - \frac{2}{\left| \mathbf{r} - \mathbf{R}(t) \right|},
      \end{equation}
      %
      where $\mathbf{R}(t) = (b,0,V t)$ is the straight-line trajectory of the projectile with velocity
      $V$ and impact parameter (distance of closest approach) $b$.
      
      The next term is the Hartree screening potential
      %
      \begin{equation} \label{eq:vh}
         v_\mathrm{H}(\mathbf{r},t) = \int \frac{n(\mathbf{r}^\prime, t)}
            {\left| \mathbf{r} - \mathbf{r}^\prime\right|} \, \mathrm{d}^3 r^\prime.
      \end{equation}
      %
      The last term is the exchange-correlation potential which encodes the complicated
      electron-electron interaction potential into the language of the non-interacting system. For
      convenience this is often further broken down into separate exchange and correlation potentials
      %
      \begin{equation} \label{eq:vxc}
         v^\sigma_\mathrm{xc} = v^\sigma_\mathrm{x} + v^\sigma_\mathrm{c}.
      \end{equation}

      Splitting $v_\mathrm{xc}$ into an exchange and correlation part facilitates the application of the
      x-only approximation where the correlation potential is taken to be zero ($v^\sigma_\mathrm{c} =
      0$). Within this approximation $v^\sigma_\mathrm{x}$ may be determined exactly via the optimized
      potential method~\cite{opm1, opm2, tdopm}. The complexity of the \textsc{opm} make it difficult
      prohibitively difficult to implement in general. As a secondary option one my instead make use of
      the Krieger-Li-Iafrate approximation~\cite{klieq, tdkli1, tdkli2} (\textsc{kli}). In many
      situations potentials generated using \textsc{kli} are numerically indistinguishable from those
      produced by the full \textsc{opm}. The success of the \textsc{kli} is due to the fact that it
      preserves several properties of the exact potential. In particular \textsc{kli} maintains the
      correct asymptotics
      %
      \begin{equation} \label{eq:asymp}
         \lim\limits_{r \rightarrow \infty} v_\mathrm{x}^\sigma (\mathbf{r}) = -\frac{1}{r}.
      \end{equation}
   
   \end{subsection}

   \begin{subsection}{The exchange potential \label{sec:xpot}}

      The one-particle density was determined by solving Eq.~\eqref{eq:tdks} using the two-center basis
      generator method~\cite{tcbgm} (\textsc{tc-bgm}). As mentioned above this relies upon the
      specification of an exchange-correlation potential. While the correlation potential may be
      ignored, that is the x-only approximation may be used (with some understanding of the
      consequences), an accurate exchange-potential is essential for a precise description of the
      He\textsuperscript{+}-He collision system. The spin polarized nature of the system, which
      emphasizes exchange effects. To this end the \textsc{kli} approximation to the \textsc{opm} was
      employed in the calculation of $v^\sigma_\mathrm{x}$.

      The ground-state density functional theory (\textsc{dft}) scheme of~\cite{diamol} has been adapted
      to calculate a time-dependent exchange potential. At any instant of time, $t$, the
      He\textsuperscript{+}-He system may be regarded as a diatomic quasi-molecule with an internuclear
      distance $R_\mathrm{int}(t) = \sqrt{b^2 + z(t)^2}$, where $b$ and $z$ are the impact parameter and
      $z$ position of the projectile as described below Eq.~\eqref{eq:hephe-ext}. If at each time-step
      of the \textsc{tc-bgm} the time-dependent Kohn-Sham orbitals, $\varphi_{\sigma j}(\mathbf{r},t)$
      are fed into the \textsc{kli} functional an exchange-potential,
      $v^{\sigma}_\mathrm{x}[\varphi_{\sigma j};t]$, may be calculated at each $t$, effectively
      giving one the time-dependent exchange-potential $v^{\sigma}_\mathrm{x}[\varphi_{\sigma j}](t)$.
      The restriction of the \textsc{kli} scheme to systems of cylindrical symmetry complicates this
      processes, in general $\varphi_{j \sigma}$ will not be forced to exhibit any specific symmetry.
      
      In order to detail a solution to the symmetry problem a more thorough description of the
      \textsc{tc-bgm} is necessary. Within the \textsc{tc-bgm} the Kohn-Sham orbitals are represented in
      the non-orthogonal basis
      %
      \begin{equation} \label{eq:bgmexp}
         \varphi_{\sigma j}(\mathbf{r},t) = \sum\limits_{c \in \{P, T\}} \sum\limits_{k, L}
                               d_{c k L}^{\sigma j}(t) \tilde{\chi}^{L}_{c k}(\mathbf{r},t),
      \end{equation}
      %
      where
      %
      \begin{equation} \label{eq:etfbasis}
         \tilde{\chi}^{L}_{ck}(\mathbf{r}),t =
            \begin{cases}
               e^{i \mathbf{v}_T \cdot \mathbf{r}} {\chi}^{L}_{c k}(\mathbf{r},t) & c = T, \\[2ex]
               e^{i \mathbf{v}_P \cdot \mathbf{r}} {\chi}^{L}_{c k}(\mathbf{r},t) & c = P,
            \end{cases}
      \end{equation}
      %
      which are the basis functions with electron translation factors (\textsc{etf}) included. The basis
      functions themselves are given by
      %
      \begin{equation} \label{eq:bgmbasis}
         \chi^{L}_{ck} (\mathbf{r},t)
         = W_P( \mathbf{r},t, \epsilon_P)^L \chi^{0}_{ck} (\mathbf{r}),
      \end{equation}
      %
      with
      \begin{equation}
         W_P (\mathbf{r},t,\epsilon_P)
         = \frac{1 - e^{-\epsilon_P|\mathbf{r}_T - \mathbf{R}(t)|}}{|\mathbf{r}_T - \mathbf{R}(t)|},
      \end{equation}
      %
      where $\mathbf{r}_T$ represents the position vector with respect to the target center.
      
      In Eq.~\eqref{eq:bgmbasis} the functions $\chi^{0}_{ck}$ are the ground-state orbitals for the
      target helium atom ($c = T$) and the projectile helium ion ($c = P$). Additional states that
      generated by a target potential operator are possible however in order to keep the number
      of states in the basis to a minimum and simplify the description only the projectile states are
      included. This simplification has proved sufficient in the past~\cite{bgm-rev}. The remaining
      regularizer is set to $\epsilon_P = 1$.

      It is clear from the above description that only the basis states corresponding s-type orbitals
      will make cylindrically symmetric contributions to the \textsc{ks}-orbitals. The simplest solution
      is to only feed the 1s contributions into the \textsc{kli} functional, more explicitly the orbital
      will take the form (leaving out \textsc{etf}'s)
      %
      \begin{equation} \label{eq:1sonly}
         \varphi_{\sigma j}^{1s} = a^{\sigma j}_T \chi^{0}_{T1} + a^{\sigma j}_P \chi^{0}_{P1}.
      \end{equation}
      %
      The coefficients are the result of projecting onto the two-dimensional subspace spanned by the
      target and projectile 1s-states
      %
      \begin{equation} \label{eq:proj}
         \hat{P} = \sum\limits_{c_1, c_2 \in \{T,P\}} \tilde{S}^{-1}_{c_1 c_2}
                                                      \ket{\chi^{0}_{c_1 1}}
                                                      \bra{\chi^{0}_{c_2 1}}
      \end{equation}
      %
      with $\tilde{S}^{-1}_{c_1 c_2}$ the inverse of the overlap matrix
      %
      \begin{equation} \label{eq:overlap2d}
         \tilde{S}_{c_1 c_2} = \braket{ \chi^{0}_{c_1 1} | \chi^{0}_{c_2 1} }.
      \end{equation}
      %
      The coefficients are then
      %
      \begin{equation} \label{eq:coef}
         \begin{split}
            a^{\sigma j}_c & = \braket{ \chi^{0}_{c 1} | \hat{P} | \varphi_{\sigma j} } \\
                           & = \sum\limits_{c_1, c_2 \in \{T,P\}} \sum\limits_{k_1, k_2 = 1}^K
                               \sum\limits_{l_1, l_2 = 1}^L \tilde{S}^{-1}_{c c_1}
                               S_{c_1 \, k_1 \, l_1 }^{c_2 \, k_2 \, l_2 } d^{j \sigma}_{c_2 k_2 l_2},
         \end{split}
      \end{equation}
      %
      with
      %
      \begin{equation} \label{eq:overlap}
         S_{c_1 \, k_1 \, l_1 }^{c_2 \, k_2 \, l_2 } =
            \braket{ \chi^{l_1}_{c_1 k_1} | \chi^{l_2}_{c_2 k_2} }
      \end{equation}
      %
      the full overlap matrix.

      Returning to the question of the \textsc{etf}, working n the rotating center of mass frame the
      \textsc{etf}'s become
      %
      \begin{subequations} \label{eq:etf}
         \begin{equation} \label{eq:etfT}
            e^{i \mathbf{v}_T \cdot \mathbf{r}} =
             e^{\frac{i v_\mathrm{rel}}{2} (x \sin \theta - z \cos \theta)},
         \end{equation}
         %
         \begin{equation} \label{eq:etfP}
            e^{i \mathbf{v}_P \cdot \mathbf{r}} =
             e^{\frac{i v_\mathrm{rel}}{2} (z \cos \theta - x \sin \theta)},
         \end{equation}
      \end{subequations}
      %
      where $v_\mathrm{rel}$ is the relative velocity of the centers and $\theta = \arctan b/z$. If we
      now introduce a two-centered coordinated system, placing the foci at the two nuclear centers
      %
      \begin{equation} \label{eq:psc}
            \left\{
            \begin{array}{l}
               x = \frac{|R|}{2} \sqrt{(\xi^2 - 1)(1 - \eta^2)} \sin \phi, \\
               y = \frac{|R|}{2} \sqrt{(\xi^2 - 1)(1 - \eta^2)} \cos \phi, \\
               z = \frac{|R|}{2} \xi \eta,
            \end{array}
            \right.
      \end{equation}
      %
      it becomes clear that the portion of the \textsc{etf}'s containing only $x$ breaks cylindrical
      symmetry.

      Two obvious solutions present themselves. First, one may simply ignore the \textsc{etf}'s
      completely. This would amount to passing the orbitals described by Eq.~\eqref{eq:1sonly} into
      the \textsc{kli} functional. Alternately the symmetry breaking portions of the \textsc{etf}'s
      may be dropped. In this case the full 1s-only \textsc{ks}-orbital becomes
      %
      \begin{equation} \label{eq:1sonlyetf}
         \tilde{\varphi}_{\sigma j}^{1s} =
                       a^{\sigma j}_T e^{- \frac{i v_\mathrm{rel} z \cos \theta}{2}} \chi^{0}_{T1}
                     + a^{\sigma j}_P  e^{\frac{i v_\mathrm{rel} z \cos \theta}{2}}\chi^{0}_{P1}.
      \end{equation}
      %
      Regardless of which option is chosen it is important that $V_\mathrm{H}$ be determined with the
      same set of orbitals used in the calculation of $v_\mathrm{x}^\sigma$, preserving the precise
      cancellation of the self interacion present in the Hartee potential.

   \end{subsection}

\end{section}

\begin{section}{Discussion \label{sec:disc}}

   \begin{figure}[htp]
      \centering
      \includegraphics[width = \linewidth]{./images/crossPlots/HepHe-111.eps}
      \caption{Total Cross section for single ionization of the target.
               Theoretical: response p\textsc{etf} (solid line), response n\textsc{etf} (dotted line),
                            no response p\textsc{etf} (dashed line), and response n\textsc{etf}
                            (dash-dotted line).
               Experimental: diamonds~\cite{Dub-89}, circles~\cite{FTFHLP-95}, and squares~\cite{DT-88}.
               \label{fig:cs111}}
   \end{figure}

   \begin{figure}[htp]
      \centering
      \includegraphics[width = \linewidth]{./images/crossPlots/HepHe-201.eps}
      \caption{Total Cross section for single ionization of the projectile.
               Theoretical: response p\textsc{etf} (solid line), response n\textsc{etf} (dotted line),
                            no response p\textsc{etf} (dashed line), and response n\textsc{etf}
                            (dash-dotted line).
               Experimental: diamonds~\cite{Dub-89}.
\label{fig:cs201}}
   \end{figure}

   \begin{figure}[htp]
      \centering
      \includegraphics[width = \linewidth]{./images/crossPlots/HepHe-120.eps}
      \caption{Total Cross section for single capture to the projectile.
               Theoretical: response p\textsc{etf} (solid line), response n\textsc{etf} (dotted line),
                            no response p\textsc{etf} (dashed line), and response n\textsc{etf}
                            (dash-dotted line).
               Experimental: diamonds~\cite{Dub-89} and circles~\cite{FTFHLP-95}.
 \label{fig:cs120}}
   \end{figure}

   \begin{figure}[htp]
      \centering
      \includegraphics[width = \linewidth]{./images/crossPlots/HepHe-012.eps}
      \caption{Total Cross section for double ionization of the target.
               Theoretical: response p\textsc{etf} (solid line), response n\textsc{etf} (dotted line),
                            no response p\textsc{etf} (dashed line), and response n\textsc{etf}
                            (dash-dotted line).
               Experimental: diamonds~\cite{Dub-89}, circles~\cite{FTFHLP-95}, and squares~\cite{DT-88}.
 \label{fig:cs012}}
   \end{figure}

   \begin{figure}[htp]
      \centering
      \includegraphics[width = \linewidth]{./images/crossPlots/HepHe-102.eps}
      \caption{Total Cross section for simultaneous single ionization of the target and projectile.
               Theoretical: response p\textsc{etf} (solid line), response n\textsc{etf} (dotted line),
                            no response p\textsc{etf} (dashed line), and response n\textsc{etf}
                            (dash-dotted line).
               Experimental: diamonds~\cite{Dub-89} and crosses~\cite{SSMSM-11}.
 \label{fig:cs102}}
   \end{figure}

   \begin{figure}[htp]
      \centering
      \includegraphics[width = \linewidth]{./images/crossPlots/HepHe-021.eps}
      \caption{Total Cross section for transfer ionization of the target.
               Theoretical: response p\textsc{etf} (solid line), response n\textsc{etf} (dotted line),
                            no response p\textsc{etf} (dashed line), and response n\textsc{etf}
                            (dash-dotted line).
               Experimental: diamonds~\cite{Dub-89} and circles~\cite{FTFHLP-95}.
 \label{fig:cs021}}
   \end{figure}

   \begin{figure}[htp]
      \centering
      \includegraphics[width = \linewidth]{./images/crossPlots/HepHe-003.eps}
      \caption{Total Cross section simultaneous double target and single projectile ionization.
               Theoretical: response p\textsc{etf} (solid line), response n\textsc{etf} (dotted line),
                            no response p\textsc{etf} (dashed line), and response n\textsc{etf}
                            (dash-dotted line).
               Experimental: diamonds~\cite{Dub-89} and crosses~\cite{SSMSM-11}.
 \label{fig:cs003}}
   \end{figure}

\end{section}

\begin{section}{Conclusions \label{sec:conc}}
\end{section}

\bibliography{hephe}

\end{document}

